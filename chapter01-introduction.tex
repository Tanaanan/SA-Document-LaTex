\chapter{บทนำ}

\section{ที่มาและความสำคัญของปัญหา}

บริษัทขนส่งน้ำมันในเครือ PT ดำเนินการจัดส่งน้ำมันจากแหล่งกักเก็บไปยังจุดหมายปลายทางต่างๆ โดยใช้รถบรรทุกน้ำมัน ซึ่งมีขั้นตอนการทำงานแบบดั้งเดิมคือ พนักงานจัดส่งจะวางแผนการขนส่งผ่านไฟล์ Excel และส่งมอบงานให้พนักงานขับรถผ่านกลุ่ม LINE เพื่อให้พนักงานขับรถยืนยันการเริ่มงานและจัดส่งน้ำมันตามที่ได้รับมอบหมาย เมื่อพนักงานขับรถไปถึงปลายทางจะต้องแจ้งในกลุ่ม LINE เพื่อให้พนักงานจัดส่งติดตามสถานะและบันทึกข้อมูลเวลาที่รถไปถึงและทำเวลาได้ตามกำหนดหรือไม่ลงใน Excel หลังจากนั้นเมื่อกลับถึงจุดหลัก พนักงานจัดส่งจะบันทึกปริมาณน้ำมันที่ส่งและที่เหลือ เพื่อคำนวณและเบิกค่าเที่ยวให้แก่พนักงานขับรถ

อย่างไรก็ตามระบบการทำงานดังกล่าวมีข้อจำกัดหลายประการ โดยเฉพาะขั้นตอนการติดตามรถขนส่งที่ต้องใช้พนักงานจัดส่งคอยติดตามและกรอกข้อมูลลงใน Excel อยู่ตลอดเวลา ซึ่งอาจนำไปสู่ข้อผิดพลาดและความไม่สะดวกในการติดตาม อีกทั้งการสื่อสารผ่านกลุ่ม LINE ระหว่างพนักงานจัดส่งและพนักงานขับรถยังทำให้การติดตามข้อมูลย้อนหลังเป็นเรื่องยาก และมีข้อจำกัดด้านจำนวนสมาชิกในกลุ่ม จึงจำเป็นต้องสร้างกลุ่ม LINE เพิ่มหากมีพนักงานขนส่งจำนวนมาก

จากปัญหาที่กล่าวมาการพัฒนาระบบบริหารจัดการการขนส่งน้ำมันแบบอัตโนมัติจึงมีความสำคัญอย่างยิ่งเพื่อลดภาระงานของพนักงานจัดส่งในการติดตามรถตลอดเวลา และทำให้การสื่อสารระหว่างพนักงานเป็นระบบมากยิ่งขึ้น นอกจากนี้ระบบยังช่วยจัดการข้อมูลพนักงานขนส่ง, รถ, รายละเอียดการจัดส่ง, คำนวณค่าเที่ยว และจัดเก็บข้อมูลการขนส่งได้อย่างละเอียดทำให้เกิดประโยชน์ทั้งกับพนักงานจัดส่งที่ไม่ต้องคอยติดตามรถแต่ละคันตลอดเวลา และพนักงานขับรถที่ไม่ต้องรับส่งข้อมูลผ่านกลุ่ม LINE อีกต่อไป ซึ่งจะช่วยประหยัดทั้งเวลาในการจัดการขนส่งและการติดตามงานได้อย่างมีประสิทธิภาพ


% ปัญหาที่พบ
\section{ปัญหาที่พบ}
จากการสอบถามผู้ใช้งานที่เป็นทั้งพนักงานจัดส่ง และพนักงานขับรถ พบว่ามีปัญหาเกิดขึ้นดังนี้

\subsection{พนักงานจัดส่ง}

\begin{enumerate}
    \item \textbf{จ่ายงานให้พนักงานขับรถผ่านกลุ่ม LINE} : เกิดความสับสน, หางานที่ได้รับมอบหมายลำบาก และไม่สะดวกที่จะเรียกดูข้อมูลในภายหลัง
    \item \textbf{ติดตามรถขนส่งในแต่ละจุด และบันทึกลง Excel} : ใช้เวลาเยอะ และต้องบันทึกข้อมูลการขนส่งด้วยมือ
    \item \textbf{บันทึกปริมาณแก๊สที่ส่ง และเหลือลง Excel} : ต้องบันทึกด้วยมือ และพนักงานขับรถต้องบอกด้วยตัวเองทำให้เสียเวลา และเสี่ยงต่อการเกิดข้อผิดพลาด
\end{enumerate}

\subsection{พนักงานขับรถ}

\begin{enumerate}
    \item \textbf{ยืนยันการเริ่มงานผ่านกลุ่ม LINE} : ไม่สะดวกในการเรียกดูข้อมูลในภายหลัง
\end{enumerate}

% วัตถุประสงค์
\section{วัตถุประสงค์}
    โครงงานนี้มีวัตถุประสงค์เพื่อพัฒนาระบบบริหารจัดการการขนส่งแก๊สให้กับเครือ PT เพื่อสร้างระบบช่วยให้การขนส่งแก๊สที่มีการจัดส่งเป็นประจำเป็นไปได้อย่างสะดวกและมีประสิทธิภาพ โดยที่ระบบสามารถติดตามรถในแต่ละจุดได้โดยไม่จำเป็นต้องใช้พนักงานจัดส่งในการกรอกข้อมูลในทุก ๆ จุด นอกจากนี้พนักงานขับรถกดรับงานผ่าน LINE OA ที่มีการจัดเก็บประวัติทำให้มีการตรวจสอบย้อนหลังได้, ใช้ระบบในการติดตามตำแหน่ง และอัปเดตสถานะของรถแต่ละคัน ทำให้ลดเวลาในการติดตาม และกรอกข้อมูล, บันทึกปริมาณแก๊สที่ส่งที่เหลือผ่าน LINE OA ทำให้พนักงานขับรถบันทุกได้ด้วยตัวเอง โดยที่ไม่ต้องเดินทางไปแจ้งพนักงานจัดส่ง และจัดเก็บข้อมูลลงในฐานข้อมูลในแต่ละขั้นตอน ทำให้พนักงานจัดส่งสามารถเอาข้อมุลไปใช้ได้เลย และแก้ปัญหาข้อผิดพลาดในการบันทึกข้อมูลได้

\section{อุปกรณ์ที่ใช้}
\begin{enumerate}
    \item เว็บไซต์ Exceildraw, Draw.io ใช้สำหรับวาด Diagram
    \item เว็บไซต์ Figma ใช้สำหรับออกแบบหน้าต่าง Interface ของระบบ
    \item เว็บไซต์ Vercel ใช้สำหรับทดสอบใช้งานบน Public Domain
    \item โปรแกรม Visual Studio Code ใช้สำหรับพัฒนาระบบ
    \item ภาษา TypeScript ใช้สำหรับพัฒนาระบบ
    \item ภาษา Next.js ใช้สำหรับพัฒนาระบบ
    \item ฐานข้อมูล myPHPAdmin ใช้สำหรับพัฒนาระบบ
    \item ระบบ LINE OA ใช้สำหรับพัฒนาระบบ
\end{enumerate}

\section{ขอบเขตของการทำงาน}
ระบบบริหารจัดการการขนส่งแก๊ส โดยใช้ในการจัดการขนส่งแก๊สเฉพาะในบริษัท โดยพนักงานจัดส่งสามารถมอบหมายงาน, ติดตามสถานะรถ, ติดถามสถานะแก๊ส, คำนวณค่าเที่ยว, สรุปรายละเอียดการขนส่งในแต่ละรอบได้อย่างสะดวก และพนักงานขับรถสามารถ กดรับงาน, กดเริ่มงาน, บันทึกปริมาณแก๊สที่ส่ง และกดจบงานได้โดยง่าย ทำให้การติดต่อระหว่างสองฝ่ายรวมถึงการจัดส่งแก๊สเป็นไปได้อย่างราบรื่นและมีประสิทธิภาพ

\section{ขั้นตอนการทำงาน}

\begin{enumerate}
    \item ประชุมกันในกลุ่ม เพื่อวิเคราะห์ปัญหาที่ต้องการแก้ไข
    \item สอบถามข้อมูลจากบริษัท ศึกษาระบบและสถานที่จัดส่งจริง พร้อมทั้งวางแผนขั้นตอนการทำงาน ขอบเขตการทำงน วิเคราะห์ปัญหา และออกแบบวิธีการแก้ไขของปัญหา
    \item ออกแบบ Diagram, User Requirement และฐานข้อมูล
    \item สร้างระบบฐานข้อมูล และพัฒนาระบบบริหารจัดการการขนส่งแก๊ส
    \item ทดสอบ ปรับปรุง และแก้ไขปัญหาที่เกิดขึ้นในระบบ
    \item จัดทำรายงาน และสื่อประกอบการนำเสนอ
\end{enumerate}

\section{ประโยชน์ที่คาดว่าจะได้รับ}
\begin{enumerate}
    \item ลดความยุ่งยาก และข้อผิดพลาดในการจัดส่งแก๊ส
    \item ลดการทำงานของพนักงานจัดส่ง
    \item ระบบการจัดการการขนส่งแก๊ส ที่ครบในระบบเดียว
    \item ลดขั้นตอนในการดำเนินงานของพนักงาน
    \item การขนส่งแก๊สเป็นไปได้อย่างรวดเร็ว และมีประสิทธิภาพ
\end{enumerate}

