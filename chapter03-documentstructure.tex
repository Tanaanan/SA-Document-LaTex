\chapter{โครงสร้างเอกสาร}
\label{chapter:DocumentStructure}

\section{โครงสร้างของแฟ้มต้นฉบับ}
ในเอกสารต้นฉบับสำหรับการเรียงพิมพ์ด้วย \LaTeX~นั้น นอกจากจะมีข้อความที่ต้องการให้แสดงผลแล้ว ยังมีคำสั่งพิเศษต่าง ๆ ที่ใช้ในการเรียงพิมพ์ด้วย
\begin{figure}[h]
\begin{lstlisting}
\documentclass[a4paper,twocolumn]{article}
% preamble
\usepackage{fontspec}

\title{My first article}
\author{My Name}

%document
\begin{document}
	\maketitle
	
	My text
\end{document}
\end{lstlisting}
	\caption{โครงสร้างแฟ้มต้นฉบับเอกสาร \LaTeX}
	\label{Fig:Structure}
\end{figure}

โดยทั่วไป แฟ้มต้นฉบับจะประกอบด้วยสองส่วนหลัก คือ ส่วนก่อนเริ่มต้นเอกสาร (preamble) และส่วนที่เป็นเอกสาร (document) ดังรูปที่~\ref{Fig:Structure}
โดย
\begin{itemize}
	\item ส่วนที่เป็นเอกสารคือทั้งหมดที่อยู่ระหว่าง \lstinline|\begin{document}| และ \lstinline|\end{document}| ซึ่งคือบรรทัดที่ 8-13
	\item และสิ่งที่อยู่ก่อนหน้านั้นทั้งหมด (บรรทัดที่ 1-7) เรียกว่า ส่วนก่อนเริ่มต้นเอกสาร
\end{itemize}

ส่วนก่อนเริ่มต้นเอกสารจะเป็นตัวกำหนดการตั้งค่าเอกสารที่ไม่ขึ้นกับเนื้อหาเฉพาะจุด เช่น แบบอักษร ตำแหน่งของหัวกระดาษและท้ายกระดาษ ขนาดกระดาษ ซึ่งค่าเหล่านี้ใช้ร่วมกันทั้งเอกสาร
และเป็นส่วนสำหรับประกาศการเรียกใช้แพ็คเกจต่าง ๆ ที่จะถูกเรียกใช้ในเอกสาร เพื่อให้โปรแกรมเรียงพิมพ์สามารถเรียกใช้ได้ถูกต้องเมื่อทำงาน

\section{การจัดหน้าทั่วไป}
\subsection{การใช้รูปแบบเอกสารต่าง ๆ ที่กำหนดไว้แล้ว}
ป้ายระบุ \lstinline|\documentclass| ที่บรรทัดแรกของเอกสารจะเป็นตัวกำหนดคลาส (class) หรือรูปแบบสำหรับการเรียงพิมพ์
คลาสมาตรฐานที่มีมาให้เมื่อลงโปรแกรมได้แก่ article, report, plain
แต่วารสารและการประชุมวิชาการต่าง ๆ รวมถึงวิทยานิพนธ์ของมหาวิทยาลัยจะกำหนดรูปแบบเฉพาะในการเรียงพิมพ์เอาไว้ เช่น ieeetran, amsart

รูปแบบการเรียงพิมพ์ เช่น การตั้งกั้นหน้า การขึ้นย่อหน้า ขนาดตัวอักษร รูปแบบตาราง การใส่ชื่อรูปภาพและตาราง จะกำหนดไว้ในแฟ้มที่มีนามสกุล cls (class) ซึ่งสิ่งตีพิมพ์ต่าง ๆ จะสร้างมาให้แล้วและให้ดาวน์โหลดไปใช้ได้
วิธีการใช้งานรูปแบบเหล่านี้ทำได้โดยกำหนดป้ายระบุ documentclass ให้เป็นชื่อรูปแบบแฟ้มที่มีนามสกุล cls นั้นๆ


\section{โครงสร้างเนื้อหาในเอกสาร}
ตอนต้นของส่วนเนื้อหา (document) มักจะเป็นข้อมูลชื่อเรื่อง (title) ชื่อผู้แต่ง (author) วันที่ (date) ของเอกสารนั้นๆ template ส่วนมากจะให้ผู้เขียนเอกสารระบุข้อมูลเหล่านี้ แล้วนำไปเรียงพิมพ์ให้ด้วยคำสั่ง \lstinline|\maketitle|

ถัดจากส่วนหัวข้อ มักจะเป็นส่วนบทคัดย่อ (abstract) ซึ่งระบุด้วย environment abstract

และหลังจากนั้น จึงเป็นเนื้อหาจริงๆ หากเป็นหนังสือหรือรายงานจะแบ่งเป็นบท (chapter) หากเป็นบทความวิจัย มักจะเริ่มต้นที่ตอน (section) และแยกย่อยลงไปเป็นตอนย่อย (subsection) และอาจะมีตอนย่อยลงไปอีก (subsubsection)
เราสามารถกำหนดข้อความต่างๆให้เป็นชื่อเรื่อง ชื่อผู้แต่ง ชื่อบท ชื่อตอน ได้ โดยใช้ป้ายระบุแบบกำหนดขอบเขต ดังนี้

\begin{multicols}{2}
	\begin{itemize}
		\item \lstinline|\title{...}| 
		\item \lstinline|\author{...}| 
		\item \lstinline|\chapter{...}| 
		\item \lstinline|\section{...}| 
		\item \lstinline|\subsection{...}| 
		\item \lstinline|\subsubsection{...}| 
	\end{itemize}
\end{multicols}
ลำดับเลขของส่วนต่าง ๆ จะถูกกำหนดขณะเรียงพิมพ์เรียงลำดับตามเนื้อหา ไม่จำเป็นต้องกำหนดตัวเลขเอง

สำหรับการทำสารบัญนั้น หากใช้โครงสร้างเนื้อหาด้วยป้ายระบุต่าง ๆ แล้ว  การสร้างสารบัญทำได้โดยป้ายระบุ \lstinline|\tableofcontents| รูปแบบของหน้าสารบัญถูกกำหนดโดยคลาส
ส่วนสารบัญภาพและตารางทำได้โดยเติมป้ายกำกับภาพและตารางทั้งหมด แล้วสร้างโดยใช้ป้ายระบุ \lstinline|\listoftables| และ \lstinline|\listoffigures| ตามลำดับ 

\section*{หัวข้อไม่นับเลข}
ทั้งนี้ เราสามารถระบุโครงสร้างแต่ไม่นับเลขได้ด้วยการใส่ * หลังป้ายกำกับคำสั่งระบุระดับ หัวข้อระดับนี้จะไม่ปรากฏในสารบัญ

จะเห็นได้ว่า หัวข้อนี้มีรูปแบบเดียวกับ section เว้นแต่ว่าไม่มีเลขลำดับ และหัวข้อถัดไป จะนับเลขต่อจากหัวข้อก่อนหน้า โดยนับข้ามหัวข้อนี้ไป

\section{โครงสร้างเนื้อหาในเอกสาร (ต่อ)}
\subsection{โครงสร้างหนังสือ}
หนังสือมักจะแบ่งส่วนเอกสารออกเป็นส่วนหน้า (frontmatter) ส่วนเนื้อหา (mainmatter) และส่วนหลัง (backmatter) การแบ่งเหล่านี้มักจะอำนวยความสะดวกในการนับเลขหน้าเพื่อไปใส่ในสารบัญด้วย โดยส่วนหน้าและส่วนหลังมักจะไม่นับเลขหน้า หรือหากนับ ก็จะนับแยกจากส่วนเนื้อหา และใช้เลขโรมัน

ข้อมูลที่มักจะอยู่ในส่วนหน้าของหนังสือ ได้แก่
\begin{enumerate}
	\item หน้าปก
	\item คำนำ (preface)
	\item คำอุทิศ (acknowledgement)
	\item สารบัญ (tableofcontents)
	\item สารบัญรูป (listoffigures)
	\item สารบัญตาราง (listoftables)
\end{enumerate}
การเรียงลำดับข้อมูลนี้สามารถปรับได้ตามความต้องการ เช่นในหนังสือหรือตำราเรียน มักจะใส่คำนำไว้หลังจากสารบัญ เป็นต้น

ถัดจากส่วนเนื้อหา สามารถมีส่วนภาคผนวก ก่อนจะเป็นส่วนท้าย ซึ่งเป็นบรรณานุกรม ดรรชนี และข้อมูลเกี่ยวกับผู้เขียน

