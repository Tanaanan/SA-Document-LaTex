\chapter{การจัดรูปแบบ}

\section{ขนาดและรูปแบบตัวอักษร}
\label{Subsect:Font}
\begin{table}[h]
	\center
	\caption{การกำหนดขนาดของตัวอักษร}
	\begin{tabular}{|c|c|}
		\hline
		คำสั่ง	& ผลลัพธ์ \\
		\hline
		\lstinline|\tiny|	& \tiny text ข้อความ  \\
		\lstinline|\scriptsize|	& \scriptsize text ข้อความ \\
		\lstinline|\footnotesize|	& \footnotesize text ข้อความ \\
		\lstinline|\small|	& \small text ข้อความ \\
		\lstinline|\normalsize|	& {\normalsize text ข้อความ} ขนาดปกติ \\
		\lstinline|\large|	& \large text ข้อความ \\
		\lstinline|\Large|	& \Large text ข้อความ \\
		\lstinline|\huge|	& \huge text ข้อความ \\
		\lstinline|\Huge|	& \Huge text ข้อความ \\
		\hline
	\end{tabular}
	\label{Table:FontSize}
\end{table}

ในการพิมพ์ข้อความอาจมีความจำเป็นต้อง {\Large เพิ่ม} หรือ {\scriptsize ลด} ขนาดของตัวอักษรในบางจุด การสร้างเอกสารโดย \LaTeX~จะมีป้ายระบุเพื่อใช้กำหนดขนาดสัมพัทธ์กับขนาดปกติโดยใช้ป้ายระบุดังแสดงในตารางที่ \ref{Table:FontSize}
โดยที่ป้ายระบุเพื่อกำหนดขนาดตัวอักษรเหล่านี้เป็นป้ายระบุแบบไม่ใช้ตัวแปร ซึ่งจะเรียงพิมพ์ด้วยคำสั่งนี้ไปจนกว่าจะหมดขอบเขตของข้อความ หากเราต้องการจำกัดขอบเขตของป้ายระบุสามารถทำได้โดยใส่วงเล็บปีกกาครอบทั้งป้ายระบุและข้อความเพื่อกำหนดขอบเขตบล็อกที่จะให้ป้ายระบุนั้นมีผลในการเรียงพิมพ์

\begin{table}
	\center
	\caption{การจัดรูปแบบตัวอักษร}
	\begin{tabular}{|c|c|}
		\hline
		คำสั่ง		& ผลลัพธ์ \\
		\hline
		\lstinline|\textit{...}| 	& \textit{ตัวเอียง} \\
		\lstinline|\textbf{...}| 	& \textbf{ตัวหนา} \\
		\lstinline|\underline{...}| 	& \underline{ขีดเส้นใต้} \\
		\lstinline|\sout{...}| 	& \sout{ขีดกลาง} \\
		\hline
	\end{tabular}
	\label{Table:TextFormat}
\end{table}

\textbf{การทำตัวหนา} \textit{ตัวเอียง} \underline{ขีดเส้นใต้} \sout{ขีดกลาง} (สำหรับขีดกลางต้องเรียกใช้แพ็คเกจ ulem) ใช้ป้ายระบุดังที่สรุปอยู่ในตาราง \ref{Table:TextFormat} การจัดรูปแบบตัวอักษร\textbf{\underline{\textit{ผสม}}} กันหลายรูปแบบสามารถทำได้ด้วยการใช้ป้ายระบุซ้อนกัน

โดยปกติเอกสาร \LaTeX~จะใช้ฟอนต์เดียวกันทั้งเอกสาร หากไม่ได้กำหนดฟอนต์เป็นพิเศษ ตัวอักษรทั้งหมดจะใช้ฟอนต์ Computer Modern
การเปลี่ยนฟอนต์เฉพาะบางส่วนของเอกสารทำได้หลายวิธี หากใช้ Xe\TeX/Xe\LaTeX~ซึ่งสนับสนุน Unicode การกำหนดฟอนต์จะใช้แพ็คเกจ fontspec แล้วระบุตัวแปรเป็น font ที่ต้องการ เช่น {\fontspec{TeX Gyre Heros}TeX Gyre Heros} หรือ {\fontspec{TeX Gyre Cursor}TeX Gyre Heros}
ฟอนต์ที่สามารถใช้ได้คือ True Type Font หรือ OpenType font ใด ๆ ที่ลงไว้ในเครื่อง สำหรับฟอนต์มาตรฐานของเอกสารราชการไทยในปัจจุบัน (พ.ศ. 2557) คือ TH Sarabun New 
การกำหนดฟอนต์หลักจึงทำได้ดังนี้
\clearpage
\begin{lstlisting}[numbers=none]
\usepackage{fontspec}
\setmainfont{TH Sarabun New}
\end{lstlisting}

เอกสารนี้ใช้คลาส memoir และกำหนดฟอนต์หลักเป็น TeX Gyre Thermes เทียบเท่า Times New Roman ใน Windows ขนาดของฟอนต์ปกติ 12 pt
และกำหนดขนาดฟอนต์ภาษาไทยให้ขนาดอักษรปกติเทียบเท่ากับอักษรตัวพิมพ์ใหญ่ในภาษาอังกฤษ

\section{การจัดเรียงข้อความ}
หากไม่ได้ตั้งค่าใดเป็นพิเศษ ข้อความจะถูกจัดให้ชิดทั้งซ้ายและขวาเสมอ (fully justified)
\begin{center}การจัดข้อความให้อยู่กึ่งกลางพื้นที่\end{center}
\begin{flushright}ชิดขวา\end{flushright}
\begin{flushleft}หรือบังคับให้ชิดซ้าย\end{flushleft}
ทำได้โดยคำสั่งต่อไปนี้
\begin{itemize}
	\item กึ่งกลาง
	\begin{itemize}[label=$\circ$]
		\item \lstinline|\centering| หรือ
		\item \lstinline|\begin{center} ... \end{center}|
	\end{itemize}
	\item ชิดขวา
	\begin{itemize}[label=$\circ$]
		\item \lstinline|\raggedleft| หรือ
		\item \lstinline|\begin{flushright}...\end{flushright}|
	\end{itemize}
	\item ชิดซ้าย
	\begin{itemize}[label=$\circ$]
		\item \lstinline|\raggedright| หรือ
		\item \lstinline|\begin{flushleft}...\end{flushleft}|
	\end{itemize}
\end{itemize}

ปัญหาหนึ่งที่พบกับการใช้ \LaTeX~ในภาษาไทยคือการตัดคำเพื่อจัดหน้าทำได้ไม่ดีนัก บางกรณีอาจมีคำที่ล้นขอบขวาเนื่องจากโปรแกรมไม่สามารถตัดคำได้ ปัญหานี้ยังไม่มีวิธีการแก้ไขที่ถาวร ผู้ใช้จำเป็นต้องเลือกเว้นวรรคข้อความ หรือปรับเนื้อหาข้อความเองเพื่อให้โปรแกรมพยายามเรียงพิมพ์ได้อย่างเหมาะสม

การจัดชิดซ้ายและขวาในภาษาอังกฤษนั้น \LaTeX~สามารถตัดกึ่งกลางคำโดยใช้ - (hyphen) ได้โดยอัตโนมัติ และทำได้ค่อนข้างดี ซึ่งปกติแล้วการเว้นช่องไฟต่าง ๆ เพื่อจัดหน้าในภาษาอังกฤษด้วย \LaTeX~จะสวยงามกว่าการใช้ Microsoft Word

\section{ช่องว่างแบบต่างๆ}
การเคาะวรรค ย่อหน้า หรือขึ้นบรรทัดใหม่หนึ่งบรรทัดใน \LaTeX~ นั้นมีผลเหมือนกันในการเรียงพิมพ์ คือ อนุญาตให้จัดช่องไฟกว้างแคบได้ตามความเหมาะสมเมื่อต้องการตัดคำเพื่อขึ้นบรรทัดใหม่
การเคาะวรรคหรือย่อหน้าในการพิมพ์เอกสารจึงมีไว้เพื่อให้ผู้พิมพ์อ่านข้อความได้ง่ายเท่านั้น
นอกจากนี้ การขึ้นบรรทัดใหม่บรรทัดเดียวมักมีประโยชน์ในการเขียนย่อหน้ายาวๆ
หากเป็นข้อความภาษาอังกฤษ บางครั้งจะนิยมเขียนข้อความบรรทัดละประโยค เพื่อให้สะดวกในการกลับมาหาข้อความเพื่อแก้ไขในอนาคต

แต่การจัดช่องไฟอัตโนมัติอาจทำให้การแสดงผลบางรูปแบบไม่ถูกต้อง เช่น หลังคำว่ารูปที่ ควรเคาะเพียงวรรคเดียวแล้วตามด้วยตัวเลข การบังคับให้เครื่องเว้นวรรคเดียวและห้ามตัดระหว่างคำทำได้โดยใช้เครื่องหมาย {\textasciitilde} แทนการเคาะวรรค เช่น รูปที่~1 ส่วนการบังคับให้ขึ้นบรรทัดใหม่ ใช้คำสั่ง \lstinline+\\+ \\
หรือใช้ป้ายระบุ \lstinline|\newline| \newline
ก็ได้ ทุกครั้งที่ขึ้นบรรทัดใหม่โดยใช้คำสั่งหรือป้ายระบุนี้จะไม่มีการย่อหน้า
หากต้องการให้ย่อหน้าใหม่ให้ใช้การขึ้นบรรทัดใหม่สองครั้ง \\
\indent หรือใช้ป้ายระบุ \lstinline|\indent| เพื่อบังคับให้ขึ้นย่อหน้าใหม่

\begin{comment}
	ระยะระหว่างบรรทัดจะกำหนดไว้ในคลาส แต่หากต้องการระบุเอง มีชุดคำสั่งในแพ็คเกจ setspace
	ได้แก่ \lstinline|\singlespacing|, \lstinline|\onehalfspacing|, \lstinline|\doublespacing|
	สำหรับระยะระหว่างบรรทัดหนึ่งเท่า หนึ่งเท่าครึ่ง และสองเท่า ตามลำดับ
	หากต้องการกำหนดเอง ใช้ป้ายระบุ \lstinline|\setstretch{<factor>}| แล้วกำหนด factor เป็นตัวเลขที่ต้องการ
	ในกรณีที่ไม่ต้องการใช้แพ็คเกจ อาจใช้ป้ายระบุ \lstinline|\linespread{<factor>}| ก็ได้ ซึ่งมีผลเหมือน \lstinline|\setstretch{<factor>}|
\end{comment}

บางครั้งอาจต้องบังคับเว้นระยะระหว่างย่อหน้า ระหว่างข้อความ รูปภาพ มากกว่าค่าโดยปริยายที่รูปแบบเอกสารจัดให้ คำสั่งที่ใช้ในการเว้นระยะแนวนอนคือ
\begin{lstlisting}[numbers=none]
	\hspace{<length>}
\end{lstlisting}
เช่น \hspace{10cm} อยู่ห่างจากคำก่อนหน้า 10 ซม. การกำหนดระยะสามารถระบุเป็น ซม. (cm) นิ้ว (in) พอยท์ (pt) ได้

\vspace{-1.5cm}
ส่วนคำสั่งที่ใช้ในการเว้นระยะแนวตั้งคือ
\begin{lstlisting}[numbers=none]
	\vspace{<length>}
\end{lstlisting}
จะเห็นว่าระยะที่ใช้จะเป็นลบหรือบวกก็ได้เช่นกัน


\section{ข้อย่อย}
โครงสร้างเนื้อหาอีกรูปแบบซึ่งใช้บ่อยคือข้อย่อย ใน \LaTeX~ มีโครงสร้างข้อย่อยให้เลือก 2 รูปแบบ แบบนับเลขข้อ (enumerate) และแบบไม่นับเลขข้อ (itemize) ตัวข้อย่อยแต่ละตัวจะอยู่หลังป้ายระบุ \lstinline|\item| หากต้องการซ้อนข้อย่อยก็แทรกโครงสร้างข้อย่อยลงไปในป้ายระบุ item ได้ เช่น
\begin{enumerate}
	\item ข้อแรกไม่มีข้อย่อย
	\item ข้อสองแยกเป็นข้อย่อยแบบไม่นับเลขสองข้อ ได้แก่
	\begin{itemize}
		\item bullet แรก
		\item bullet ที่สอง
	\end{itemize}
	และแบบนับเลขข้ออีกสามข้อ ดังนี้
	\begin{enumerate}
		\item ข้อสอง ข้อย่อยที่ 1
		\item ข้อสอง ข้อย่อยที่ 2
		\item ข้อสอง ข้อย่อยที่ 3
	\end{enumerate}
	\item ข้อสาม
\end{enumerate}
ทั้งนี้ เราสามารถกำหนดวิธีการนับเลข (เช่น เลขไทย เลขโรมัน อักขระ) และรูปสัญลักษณ์ของ bullet (วงกลมทึบ วงกลมโปร่ง สี่เหลี่ยม หรือสัญลักษณ์อื่นๆ) ได้เองผ่านตัวเลือกต่าง ๆ ผู้ที่สนใจสามารถไปค้นหาวิธีเพิ่มเติมได้ในอินเทอร์เน็ต
