\chapter{การแทรกรูปภาพและตาราง}
\section{การแทรกรูปภาพ}
การแทรกไฟล์รูปภาพทำได้โดยใช้แพ็คเกจ graphicx และป้ายระบุดังนี้
\begin{lstlisting}[numbers=none]
\includegraphics[options]{filename}
\end{lstlisting}

\begin{figure}[h]
	\begin{subfigure}{.5\textwidth}
		\includegraphics{Figures/Prakeaw-Pink}
		
		\lstinline|\includegraphics{Figures/Prakeaw-Pink}|
		\subcaption{รูปเต็ม}
	\end{subfigure}
	\begin{subfigure}{.5\textwidth}
		\centering	% กลางพื้นที่
		\includegraphics[scale=.5]{Figures/Prakeaw-Pink}
		
		\lstinline|\includegraphics[scale=.5]{Figures/Prakeaw-Pink}|
		\subcaption{รูปขนาดครึ่งของรูปเต็ม}
	\end{subfigure}
	
	\begin{subfigure}{.5\textwidth}
		\includegraphics[width=1cm]{Figures/Prakeaw-Pink}
		
		\lstinline|\includegraphics[width=1cm]{Figures/Prakeaw-Pink}|
		\subcaption{รูปขนาดกว้าง 1 ซม. ปรับความสูงอัตโนมัติ}
		\label{Fig:Fig1cm}
	\end{subfigure}
	\begin{subfigure}{.5\textwidth}
		\raggedleft	% ชิดขวา
		\subcaption{รูปขนาดกว้าง 2 ซม. ความสูง 1 ซม.}
		\includegraphics[width=2cm,height=1cm]{Figures/Prakeaw-Pink}
		
		\lstinline|\includegraphics[width=2cm,height=1cm]{Figures/Prakeaw-Pink}|
		\label{Fig:w2h1}
	\end{subfigure}
	\caption{ตัวอย่างการแทรกรูปภาพ}
	\label{Fig:Fig}
\end{figure}

คำสั่ง includegraphics มีตัวเลือกเพื่อปรับขนาดของภาพที่จะแทรกได้ด้วย ดังตัวอย่างในรูป~\ref{Fig:Fig} โดยปกติเรามักใช้ป้ายระบุ figure แบบเปิด/ปิดเพื่อกำหนดขอบเขตของรูป ภายในเรียกใช้คำสั่ง includegrapics แล้วตามด้วยป้ายกำกับชื่อรูปภาพเหล่านั้น เพื่อให้โปรแกรมเรียงพิมพ์สามารถแยกแยะได้ว่าป้ายกำกับเหล่านี้เป็นป้ายกำกับของรูปภาพ และสามารถนับลำดับเฉพาะรูปภาพได้

ส่วนคำบรรยายภาพจะถูกกำหนดโดยป้ายระบุ \lstinline|\caption{..}| โดยมากแล้ววารสารต่างๆ มักให้พิมพ์ข้อความบรรยายภาพไว้ใต้ภาพ แต่หากต้องการให้ข้อความบรรยายภาพอยู่เหนือภาพก็ทำได้โดยย้ายตำแหน่งของป้ายระบุ caption ไปไว้ก่อนภาพ ดังรูป~\ref{Fig:w2h1}

หากรูปมีขนาดเล็กกว่าพื้นที่หน้ากระดาษ โดยปกติรูปจะถูกวางไว้ชิดซ้าย หากต้องการจัดให้รูปอยู่กึ่งกลางหรือชิดขวาของพื้นที่ก็ใช้วิธีเช่นเดียวกับการจัดหน้าปกติ นอกจากนี้ หากต้องการใส่รูปย่อยหลายรูปในภาพเดียวกัน เราสามารถวางรูปด้วยป้ายระบุ includegraphics ต่อ ๆ กันไปได้เลย แต่หากต้องการใส่คำอธิบายภาพแยกตามรูปย่อย สามารถทำได้ผ่านการแทรกรูปย่อยด้วยแพ็คเกจ subcaption ซึ่งใช้ป้ายระบุ subfigure ที่มีโครงสร้างเหมือนป้ายระบุ figure ทุกประการ การอ้างอิงถึงรูปย่อยก็ทำเช่นเดียวกับการอ้างอิงถึงรูปปกติ เช่น รูปที่~\ref{Fig:Fig1cm} อ้างอิงถึงรูปขนาดกว้าง 1 ซม. ปรับความสูงอัตโนมัติ เป็นต้น

โปรแกรมเรียงพิมพ์จะจัดตำแหน่งของรูปภาพในเอกสารให้เอง ค่าเริ่มต้นที่ตั้งมามักเป็นการจัดให้รูปภาพอยู่บนสุดของหน้ากระดาษ แต่หากต้องการกำหนดตำแหน่งของรูปภาพให้อยู่ท้ายกระดาษ หรือ ณ บริเวณที่กำหนด เราสามารถระบุได้ผ่านตัวเลือกของป้ายระบุ figure ได้ เช่น

\begin{tabular}{lc}
	h    & ณ ตำแหน่งใกล้ ๆ นี้\\
	H    & ณ ตำแหน่งนี้ (ต้องใช้แพ็คเกจ float) \\
	t    & ด้านบนของกระดาษ \\
	b    & ด้านล่างของกระดาษ
\end{tabular}

หากไม่ใช้ป้ายระบุ figure ครอบรูปภาพ โปรแกรมจะเรียงพิมพ์รูปนั้นเสมือนเป็นอักขระหนึ่ง เช่น \includegraphics[width=.05\textwidth]{Figures/Prakeaw-Black}

\section{การสร้างตาราง}
โครงสร้างตารางอย่างง่ายใช้ป้ายระบุ tabular ดังตัวอย่างต่อไปนี้
\begin{lstlisting}[numbers=none]
\begin{tabular}{|c||lr|}
t1	& t2		& t3 \\
\hline
abcd	& defghij	& klmnop \\
\hline
\end{tabular}
\end{lstlisting}
เมื่อเรียงพิมพ์แล้ว จะให้ผลออกมาเป็น

\begin{tabular}{|c||lr|}
	t1		& t2	& t3 \\
	\hline
	abcd	& defghij	& klmnop \\
	\hline
\end{tabular}

วงเล็บปีกกาแรกถัดจาก tabular คือจำนวนและการจัดหน้าของแต่ละคอลัมน์ จากตัวอย่างข้างต้นนี้มีทั้งหมด 3 คอลัมน์ โดย c l r แทนกึ่งกลาง ชิดซ้าย และชิดขวา ตามลำดับ ตัวอักษร | แทนการขีดเส้นตั้งระหว่างคอลัมน์ ข้อความที่อยู่ระหว่าง begin และ end คือข้อมูลในตาราง ข้อความแต่ละบรรทัดคือข้อความในแต่ละแถวของตาราง คั่นระหว่างคอลัมน์ในแถวด้วยสัญลักษณ์ \& และจบแถวด้วยการสั่งขึ้นบรรทัดใหม่ การขีดเส้นนอนในตารางทำได้โดยคำสั่ง \lstinline|\hline|

ในทำนองเดียวกับรูปภาพ เรามักใช้ป้ายระบุ table แบบกำหนดขอบเขต เพื่อระบุขอบเขตของตาราง และให้โปรแกรมจัดตำแหน่งที่เหมาะสมสำหรับตารางให้เอง (หรือระบุตำแหน่งเองแบบเดียวกับรูปภาพ) รวมถึงนับลำดับการอ้างอิงแยกเป็นตารางด้วย
