\chapter{การอ้างอิง}
\section{การอ้างอิงภายในเอกสาร}
ป้ายกำกับ (label) มีโครงสร้างดังนี้
\begin{lstlisting}[numbers=none]
\label{<label name>}
\end{lstlisting}
โดย \lstinline|<label name>| คือชื่อของป้ายกำกับ การใส่ป้ายกำกับลงไปในเอกสารเพื่อใช้อ้างถึงตำแหน่งนั้น ๆ การอ้างอิงถึงบทและตอน (และตอนย่อยหากมี) ของตำแหน่งป้ายกำกับนั้นใช้ป้ายระบุ \lstinline|\ref<label name>| เช่น หากกำหนดป้ายกำกับที่นี่ \lstinline|\label{sample}| \label{sample} การอ้างถึง \lstinline|\ref{sample}| จะได้เป็น \ref{sample} ส่วนการอ้างถึงหน้าที่ตำแหน่งป้ายกำกับนั้นใช้คำสั่ง \lstinline|\pageref{<label name>}| เช่น \lstinline|\pageref{sample}| จะได้ \pageref{sample} และหากป้ายกำกับอยู่ในขอบเขตของรูปภาพและตาราง การอ้างถึงก็จะได้เลขลำดับของรูปภาพและตารางนั้นมา เช่น รูปที่ \ref{Fig:bibtex} อยู่ที่หน้า \pageref{Fig:bibtex} เป็นต้น

\section{บรรณานุกรม}
\begin{figure}
	\begin{verbatim}
@inproceedings{DBLP:conf/pricai/WanvarieTO10,
	author    = {Dittaya Wanvarie and
		Hiroya Takamura and
		Manabu Okumura},
	title     = {Active Learning for Sequence Labelling
		with Probability Re-estimation},
	booktitle = {PRICAI},
	year      = {2010},
	pages     = {681-686},
	ee = {http://dx.doi.org/10.1007/978-3-642-15246-7_69},
	crossref  = {DBLP:conf/pricai/2010},
	bibsource = {DBLP, http://dblp.uni-trier.de}
}

@book{thaibib,
	author = {{ฑิตยา หวานวารี}},
	title = {การใช้ \LaTeX สําหรับเรียงพิมพ์วิทยานิพนธ์ภาษาไทยและภาษาอังกฤษ โดยใช้รูปแบบของจุฬาลงกรณ์มหาวิทยาลัย},
	publisher = {{ภาควิชาคณิตศาสตร์และวิทยาการคอมพิวเตอร์}},
	month = {กรกฎาคม},
	year = {2557},
}
	\end{verbatim}
	\caption{ตัวอย่างข้อมูลสำหรับ BibTeX}
	\label{Fig:bibtex}
\end{figure}
Bib\TeX~เป็นโปรแกรมการจัดการรูปแบบเอกสารอ้างอิงที่ทำให้ \LaTeX~ สามารถเรียงพิมพ์ได้สะดวก หากดาวน์โหลดข้อมูลบรรณานุกรมจากเว็บของสำนักพิมพ์ ทางสำนักพิมพ์มักจะอำนวยความสะดวกด้วยการทำรายละเอียดสำหรับ Bib\TeX~ สำหรับบทความแต่ละชิ้นไว้ให้ สามารถคัดลอกไปใช้ได้ทันที เช่น รูปที่~\ref{Fig:bibtex}

ทั้งนี้ ควรตรวจแก้ไขสอบข้อมูลต่าง ๆ ให้ถูกต้องตามรูปแบบที่สิ่งพิมพ์แต่ละชนิดกำหนดด้วย เช่น ชื่อหนังสือ บางวารสารอาจบังคับให้ใช้ชื่อเต็มสำหรับการประชุมวิชาการ ในขณะที่ booktitle ในข้อมูลที่ได้มาเป็นชื่อย่อ ก็ต้องแก้เองให้ถูกต้องด้วย เป็นต้น

%อย่างไรก็ตาม Bib\TeX~ นั้นเป็นโปรแกรมเก่า ซึ่งจะมีปัญหาในการตัดบรรทัดเมื่อใช้กับข้อความที่ไม่ได้ใช้อักขระภาษาอังกฤษ ในเอกสารแก้ปัญหาแบบชั่วคราวด้วยการใช้ \% นำหน้าช่องว่าง เพื่อบอกขอบเขตของคำ ให้โปรแกรมสามารถตัดคำเพื่อขึ้นบรรทัดใหม่ได้อย่างถูกต้อง

การอ้างอิงถึงเอกสารชิ้นต่าง ๆ ในแฟ้มต้นฉบับทำได้โดยคำสั่ง
\begin{lstlisting}[numbers=none]
\cite{<cite name>}
\end{lstlisting}

และเขียนข้อมูลบรรณานุกรมของแต่ละเอกสารอ้างอิงให้ถูกต้องรวมกับไว้ในแฟ้มรายการบรรณานุกรม ซึ่งมักให้มีนามสกุล bib การระบุรูปแบบของบรรณานุให้ใช้รูปแบบที่วารสารกำหนดให้ใช้แฟ้มรูปแบบบรรณานุกรม (นามสกุล bst)

การระบุรูปแบบและแฟ้มรายการบรรณานุกรมอ้างอิงทำได้ดังนี้
\begin{lstlisting}[numbers=none]
\bibliographystyle{<bib style name>}
\bibliography{<bib name>}
\end{lstlisting}

การใช้ Bib\TeX~นั้น หากใช้โปรแกรมจำพวก IDE ในการสร้างเอกสารต้นฉบับ มักจะมีการเรียก Bib\TeX~ให้โดยอัตโนมัติ อย่างไรก็ตาม หากสั่งเรียงพิมพ์โดยตรง ต้องสั่งเรียงพิมพ์ก่อนหนึ่งครั้งเพื่อสร้างรายการบรรณานุกรม แล้วใช้ Bib\TeX~ เพื่อโยงรายการอ้างอิงเหล่านั้นกับแฟ้มข้อมูลรายการบรรณานุกรม แล้วจึงสั่งเรียงพิมพ์อีกรอบ เพื่อให้แทนค่าแต่ละตำแหน่งของคำสั่ง \textbackslash cite ด้วยข้อมูลบรรณานุกรมในเอกสารผลลัพธ์ให้ถูกต้อง

เช่นจากรูปที่~\ref{Fig:bibtex} สามารถอ้างอิงถึงได้ดังนี้
\begin{lstlisting}[numbers=none]
\cite{DBLP:conf/pricai/WanvarieTO10}
\end{lstlisting}
และผลลัพธ์ที่ได้เป็นดังนี้ \cite{DBLP:conf/pricai/WanvarieTO10} และ \lstinline|\citep{thaibib}| จะได้ผลลัพธ์เป็น \citep{thaibib}

\section{เอกสารสำหรับอ่านเพิ่มเติม}
รายละเอียดการใช้ \LaTeX~สามารถอ่านเพิ่มเติมได้ที่
\begin{itemize}
	\item \LaTeX~User's Guide and Reference Manual 
	
	\url{http://latex-project.org/guides/usrguide.pdf}
	\item \url{http://www.ctan.org/tex-archive/info/lshort/thai}
	\item \url{http://en.wikibooks.org/wiki/LaTeX}
	\item สัญลักษณ์ต่าง ๆ
	
	\url{http://www.tex.ac.uk/tex-archive/info/symbols/comprehensive/symbols-a4.pdf}
	
	\item ตัวอย่างบทความเมื่อใช้คลาสของ AMS
	
	\url{http://www.ams.org/publications/authors/tex/amslatex}
	
\end{itemize}
หรือค้นคว้าเพิ่มเติมได้จากอินเทอร์เน็ต
