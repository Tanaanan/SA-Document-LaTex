\chapter{การแทรกสมการและสัญลักษณ์คณิตศาสตร์}
การพิมพ์ข้อความหรือสัญลักษณ์คณิตศาสตร์แบ่งออกเป็น 2 รูปแบบ คือ แทรกในบรรทัดเดียวกัน (inline) เช่น $\delta^2$ หรือแยกออกมาเป็นบรรทัดต่างหากในกรณีพิมพ์สมการ เช่น
\begin{equation}
	f(x,y) = \int_a^b \frac{p(x,y)}{\ln y} dx \label{Eq:fxy}
\end{equation}
จะสังเกตได้ว่าการเรียงพิมพ์ในสภาพแวดล้อมคณิตศาสตร์จะใช้ฟอนต์ที่ต่างไปจากการเรียงพิมพ์ในสภาพแวดล้อมข้อความทั่วไป เมื่อใช้ฟังก์ชันหรือสัญลักษณ์ทางคณิตศาสตร์จึงควรใช้สภาพแวดล้อมคณิตศาสตร์เสมอ

การพิมพ์ในบรรทัดเดียวกันทำได้ด้วยการเขียนข้อความที่ต้องการให้อยู่ระหว่าง \$ และ \$ ส่วนการพิมพ์แยกบรรทัดต่างหากทำได้โดยการพิมพ์ข้อความให้อยู่ระหว่างป้ายระบุแบบเปิดปิด equation หรือ align

align จะสามารถเขียนสัญลักษณ์ต่าง ๆ ได้มากกว่า 1 บรรทัด มีได้หลายคอลัมน์ และสามารจัดเรียงตำแหน่งได้ด้วย เช่น
\begin{align}
	x    &= 2 \label{Eq:x} \\
	y    &= 3 \label{Eq:y} \\
	z    &= x \times y \nonumber \\
	&= p \label{Eq:result}s
\end{align}

การเรียงพิมพ์สมการแยกบรรทัดนั้น โปรแกรมจะสร้างหมายเลขกำกับสมการให้โดยอัตโนมัติ หากต้องการอ้างอิงถึงเลขสมการ ทำได้โดยเติมป้ายกำกับ (label) ลงไป ตั้งชื่อได้ตามต้องการ เมื่อต้องการเรียกใช้ก็เรียกใช้ป้ายกำกับนั้นๆ โดยใช้ป้ายระบุ \lstinline|\ref{<label name>}| เช่น เราอ้างอิงถึงสมการข้างต้นได้ด้วย (\ref{Eq:fxy}) คำแนะนำในการตั้งชื่อป้ายระบุคือควรตั้งให้สื่อถึงข้อความส่วนนั้น
หากไม่ต้องการให้มีเลขสมการ ให้เติม * ลงไปข้างหลัง equation เช่น

\begin{equation*}
	\mathbf{X}_k = \sum_{n=0}^{N-1} x_n \cdot e^{-i 2 \pi k n / N}, k \in \mathbb{Z}
\end{equation*}

การเขียนเมตริกซ์ การกำหนดนิยาม ทฤษฎีบท และอื่นๆ สามารถค้นคว้าเพิ่มเติมได้จากอินเทอร์เน็ต