\documentclass[a4paper,12pt]{memoir}

\usepackage{graphicx}
\usepackage{afterpage}

\XeTeXlinebreaklocale "th"
\XeTeXlinebreakskip = 0pt plus 0pt

\usepackage{fontspec}
\defaultfontfeatures{Mapping=tex-text}
\setmainfont{TeX Gyre Termes}   % Free Times
\setsansfont{TeX Gyre Heros}                % Free Helvetica
\setmonofont{TeX Gyre Cursor}               % Free Courier

\newfontfamily\thaifont
[Scale=MatchUppercase,Mapping=tex-text]{Laksaman}
\newenvironment{thailang}
{\thaifont}
{}

\usepackage[Latin,Thai]{ucharclasses}
\setTransitionTo{Thai}{\begin{thailang}}
\setTransitionFrom{Thai}{\end{thailang}}

\usepackage{polyglossia}          
\setdefaultlanguage{english}
\setotherlanguage{thai}

\DisemulatePackage{setspace}
\usepackage{setspace}
\onehalfspacing
\sloppy

\def\thaialph#1{\expandafter\thalph\csname c@#1\endcsname}
\def\thalph#1{%
    \ifcase#1\or ก\or ข\or ค\or ง\or จ\or ฉ\or ช\or ซ\or
    ฌ\or ญ\or ฎ\or ฏ\or ฐ\or ฑ\or ฒ\or ณ\or ด\or ต\or ถ\or ท\or ธ\or น\or
    บ\or ป\or ผ\or ฝ\or พ\or ฟ\or ภ\or ม\or ย\or ร\or ฤ\or ล\or ฦ\or ว\or
    ศ\or ษ\or ส\or ส\or ห\or ฬ\or อ\else ฮ\else\xpg@ill@value{#1}{thalph}\fi}

\def\thainum#1{\expandafter\thainumber\csname c@#1\endcsname}
\def\thainumber#1{%
    \thaidigits{\number#1}%
}
\def\thaidigits#1{\expandafter\thdigits #1@}
\def\thdigits#1{%
    \ifx @#1% then terminate
    \else
    \ifx0#1๐\else\ifx1#1๑\else\ifx2#1๒\else\ifx3#1๓\else\ifx4#1๔\else\ifx5#1๕\else\ifx6#1๖\else\ifx7#1๗\else\ifx8#1๘\else\ifx9#1๙\else#1\fi\fi\fi\fi\fi\fi\fi\fi\fi\fi
    \expandafter\thdigits
    \fi
}

\usepackage{graphicx}           % for includegraphics
\usepackage{url}                % for \url
\usepackage{listings}           % for psuedocode, lines of codes
\usepackage{ulem}               % for strikethrough \sout
\usepackage{multicol}           % for multicols environment
\usepackage{enumitem}           % for list labels
\usepackage[labelformat=simple]{subcaption}
\usepackage{verbatim}           % for long codes
\usepackage{natbib}             % for bibliography
\usepackage{amsmath}            % for align equation
\usepackage{amssymb}            % for math symbols
\usepackage[hidelinks]{hyperref} % Corrected link style
\setcounter{tocdepth}{2} % Add this line to show subsections in the ToC
\maxsecnumdepth{subsection} % This will enable subsection numbering
\usepackage[
type={CC},
modifier={by-nc-sa},
version={3.0},
]{doclicense}

% listings format
\lstset{
    basicstyle=\ttfamily,
    numbers=left,
    numberstyle=\small,
    xleftmargin=.1\linewidth,
    frame=single,
    captionpos=b,
    columns=fullflexible,
    breaklines=true,
    keepspaces=true,
    showstringspaces=false,
}

\begin{document}

\begin{titlingpage}
    \centering
    \includegraphics[width=4.5cm]{./Figures/logo_ku_th.png}
    
    \vspace{0.5cm}
    
    {\Large \bfseries \begin{thailang}รายงาน\end{thailang}}
    
    \vspace{0.2cm}
    
    {\Large \bfseries \begin{thailang}เรื่อง ระบบบริหารจัดการการขนส่งแก๊ส\end{thailang}}
    
    \vspace{1.5cm}
    
    \textbf{\begin{thailang}เสนอ\end{thailang}}
    
    \vspace{0.2cm}
    
    {\begin{thailang}อาจารย์สมโชค เรืองอิทธินันท์\end{thailang}}
    
    \vspace{1.5cm}
    
    {\bfseries \begin{thailang}คณะผู้จัดทำ\end{thailang}}
    
    \vspace{0.5cm}
    
    \begin{tabular}{ll}
        \begin{thailang}นาย ธนอนันท์ เฉลิมพันธ์\end{thailang} & \begin{thailang}รหัสนิสิต 6610402078\end{thailang} \\
        \begin{thailang}นาย รักษิต รุ่งรัตนไชย\end{thailang} & \begin{thailang}รหัสนิสิต 6610402205\end{thailang} \\
        \begin{thailang}นาย นรากร ธนาพรภักดี\end{thailang} & \begin{thailang}รหัสนิสิต 6610405905\end{thailang} \\
    \end{tabular}

    \vspace{1.5cm}
    
    {\begin{thailang}รายงานนี้เป็นส่วนหนึ่งของรายวิชา 01418321 การวิเคราะห์และออกแบบระบบ\end{thailang}}
    
    \vspace{0.2cm}
    
    {\begin{thailang}ภาคเรียนที่ 1 ปีการศึกษา 2568\end{thailang}}
    
    \vspace{0.2cm}
    
    {\begin{thailang}คณะวิทยาศาสตร์ ภาควิชาวิทยาการคอมพิวเตอร์\end{thailang}}
    
    \vspace{0.2cm}
    
    {\begin{thailang}มหาวิทยาลัยเกษตรศาสตร์ บางเขน\end{thailang}}
    
    \vfill
\end{titlingpage}

\frontmatter
\captionsthai

\tableofcontents
\clearpage

\listoffigures

\clearpage
\listoftables 
    
\mainmatter
% \chapter*{เกี่ยวกับเอกสาร}
% เอกสารนี้ร่างขึ้นโดยอาศัยเนื้อหาจาก \url{https://en.wikibooks.org/wiki/LaTeX}
% รหัสต้นฉบับของเอกสารนี้อยู่ที่
% \url{https://www.overleaf.com/read/cgrvbrzpxrtm} 

% ผู้ที่สนใจสามารถนำไปดัดแปลงใช้ประโยชน์ได้โดยต้องแสดงที่มา และอนุญาตให้ใช้สิทธิในแบบเดียวกัน

% \doclicenseThis

\chapter{บทนำ}

\section{ที่มาและความสำคัญของปัญหา}

บริษัทขนส่งน้ำมันในเครือ PT ดำเนินการจัดส่งน้ำมันจากแหล่งกักเก็บไปยังจุดหมายปลายทางต่างๆ โดยใช้รถบรรทุกน้ำมัน ซึ่งมีขั้นตอนการทำงานแบบดั้งเดิมคือ พนักงานจัดส่งจะวางแผนการขนส่งผ่านไฟล์ Excel และส่งมอบงานให้พนักงานขับรถผ่านกลุ่ม LINE เพื่อให้พนักงานขับรถยืนยันการเริ่มงานและจัดส่งน้ำมันตามที่ได้รับมอบหมาย เมื่อพนักงานขับรถไปถึงปลายทางจะต้องแจ้งในกลุ่ม LINE เพื่อให้พนักงานจัดส่งติดตามสถานะและบันทึกข้อมูลเวลาที่รถไปถึงและทำเวลาได้ตามกำหนดหรือไม่ลงใน Excel หลังจากนั้นเมื่อกลับถึงจุดหลัก พนักงานจัดส่งจะบันทึกปริมาณน้ำมันที่ส่งและที่เหลือ เพื่อคำนวณและเบิกค่าเที่ยวให้แก่พนักงานขับรถ

อย่างไรก็ตามระบบการทำงานดังกล่าวมีข้อจำกัดหลายประการ โดยเฉพาะขั้นตอนการติดตามรถขนส่งที่ต้องใช้พนักงานจัดส่งคอยติดตามและกรอกข้อมูลลงใน Excel อยู่ตลอดเวลา ซึ่งอาจนำไปสู่ข้อผิดพลาดและความไม่สะดวกในการติดตาม อีกทั้งการสื่อสารผ่านกลุ่ม LINE ระหว่างพนักงานจัดส่งและพนักงานขับรถยังทำให้การติดตามข้อมูลย้อนหลังเป็นเรื่องยาก และมีข้อจำกัดด้านจำนวนสมาชิกในกลุ่ม จึงจำเป็นต้องสร้างกลุ่ม LINE เพิ่มหากมีพนักงานขนส่งจำนวนมาก

จากปัญหาที่กล่าวมาการพัฒนาระบบบริหารจัดการการขนส่งน้ำมันแบบอัตโนมัติจึงมีความสำคัญอย่างยิ่งเพื่อลดภาระงานของพนักงานจัดส่งในการติดตามรถตลอดเวลา และทำให้การสื่อสารระหว่างพนักงานเป็นระบบมากยิ่งขึ้น นอกจากนี้ระบบยังช่วยจัดการข้อมูลพนักงานขนส่ง, รถ, รายละเอียดการจัดส่ง, คำนวณค่าเที่ยว และจัดเก็บข้อมูลการขนส่งได้อย่างละเอียดทำให้เกิดประโยชน์ทั้งกับพนักงานจัดส่งที่ไม่ต้องคอยติดตามรถแต่ละคันตลอดเวลา และพนักงานขับรถที่ไม่ต้องรับส่งข้อมูลผ่านกลุ่ม LINE อีกต่อไป ซึ่งจะช่วยประหยัดทั้งเวลาในการจัดการขนส่งและการติดตามงานได้อย่างมีประสิทธิภาพ


% ปัญหาที่พบ
\section{ปัญหาที่พบ}
จากการสอบถามผู้ใช้งานที่เป็นทั้งพนักงานจัดส่ง และพนักงานขับรถ พบว่ามีปัญหาเกิดขึ้นดังนี้

\subsection{พนักงานจัดส่ง}

\begin{enumerate}
    \item \textbf{จ่ายงานให้พนักงานขับรถผ่านกลุ่ม LINE} : เกิดความสับสน, หางานที่ได้รับมอบหมายลำบาก และไม่สะดวกที่จะเรียกดูข้อมูลในภายหลัง
    \item \textbf{ติดตามรถขนส่งในแต่ละจุด และบันทึกลง Excel} : ใช้เวลาเยอะ และต้องบันทึกข้อมูลการขนส่งด้วยมือ
    \item \textbf{บันทึกปริมาณแก๊สที่ส่ง และเหลือลง Excel} : ต้องบันทึกด้วยมือ และพนักงานขับรถต้องบอกด้วยตัวเองทำให้เสียเวลา และเสี่ยงต่อการเกิดข้อผิดพลาด
\end{enumerate}

\subsection{พนักงานขับรถ}

\begin{enumerate}
    \item \textbf{ยืนยันการเริ่มงานผ่านกลุ่ม LINE} : ไม่สะดวกในการเรียกดูข้อมูลในภายหลัง
\end{enumerate}

% วัตถุประสงค์
\section{วัตถุประสงค์}
    โครงงานนี้มีวัตถุประสงค์เพื่อพัฒนาระบบบริหารจัดการการขนส่งแก๊สให้กับเครือ PT เพื่อสร้างระบบช่วยให้การขนส่งแก๊สที่มีการจัดส่งเป็นประจำเป็นไปได้อย่างสะดวกและมีประสิทธิภาพ โดยที่ระบบสามารถติดตามรถในแต่ละจุดได้โดยไม่จำเป็นต้องใช้พนักงานจัดส่งในการกรอกข้อมูลในทุก ๆ จุด นอกจากนี้พนักงานขับรถกดรับงานผ่าน LINE OA ที่มีการจัดเก็บประวัติทำให้มีการตรวจสอบย้อนหลังได้, ใช้ระบบในการติดตามตำแหน่ง และอัปเดตสถานะของรถแต่ละคัน ทำให้ลดเวลาในการติดตาม และกรอกข้อมูล, บันทึกปริมาณแก๊สที่ส่งที่เหลือผ่าน LINE OA ทำให้พนักงานขับรถบันทุกได้ด้วยตัวเอง โดยที่ไม่ต้องเดินทางไปแจ้งพนักงานจัดส่ง และจัดเก็บข้อมูลลงในฐานข้อมูลในแต่ละขั้นตอน ทำให้พนักงานจัดส่งสามารถเอาข้อมุลไปใช้ได้เลย และแก้ปัญหาข้อผิดพลาดในการบันทึกข้อมูลได้

\section{อุปกรณ์ที่ใช้}
\begin{enumerate}
    \item เว็บไซต์ Exceildraw, Draw.io ใช้สำหรับวาด Diagram
    \item เว็บไซต์ Figma ใช้สำหรับออกแบบหน้าต่าง Interface ของระบบ
    \item เว็บไซต์ Vercel ใช้สำหรับทดสอบใช้งานบน Public Domain
    \item โปรแกรม Visual Studio Code ใช้สำหรับพัฒนาระบบ
    \item ภาษา TypeScript ใช้สำหรับพัฒนาระบบ
    \item ภาษา Next.js ใช้สำหรับพัฒนาระบบ
    \item ฐานข้อมูล myPHPAdmin ใช้สำหรับพัฒนาระบบ
    \item ระบบ LINE OA ใช้สำหรับพัฒนาระบบ
\end{enumerate}

\section{ขอบเขตของการทำงาน}
ระบบบริหารจัดการการขนส่งแก๊ส โดยใช้ในการจัดการขนส่งแก๊สเฉพาะในบริษัท โดยพนักงานจัดส่งสามารถมอบหมายงาน, ติดตามสถานะรถ, ติดถามสถานะแก๊ส, คำนวณค่าเที่ยว, สรุปรายละเอียดการขนส่งในแต่ละรอบได้อย่างสะดวก และพนักงานขับรถสามารถ กดรับงาน, กดเริ่มงาน, บันทึกปริมาณแก๊สที่ส่ง และกดจบงานได้โดยง่าย ทำให้การติดต่อระหว่างสองฝ่ายรวมถึงการจัดส่งแก๊สเป็นไปได้อย่างราบรื่นและมีประสิทธิภาพ

\section{ขั้นตอนการทำงาน}

\begin{enumerate}
    \item ประชุมกันในกลุ่ม เพื่อวิเคราะห์ปัญหาที่ต้องการแก้ไข
    \item สอบถามข้อมูลจากบริษัท ศึกษาระบบและสถานที่จัดส่งจริง พร้อมทั้งวางแผนขั้นตอนการทำงาน ขอบเขตการทำงน วิเคราะห์ปัญหา และออกแบบวิธีการแก้ไขของปัญหา
    \item ออกแบบ Diagram, User Requirement และฐานข้อมูล
    \item สร้างระบบฐานข้อมูล และพัฒนาระบบบริหารจัดการการขนส่งแก๊ส
    \item ทดสอบ ปรับปรุง และแก้ไขปัญหาที่เกิดขึ้นในระบบ
    \item จัดทำรายงาน และสื่อประกอบการนำเสนอ
\end{enumerate}

\section{ประโยชน์ที่คาดว่าจะได้รับ}
\begin{enumerate}
    \item ลดความยุ่งยาก และข้อผิดพลาดในการจัดส่งแก๊ส
    \item ลดการทำงานของพนักงานจัดส่ง
    \item ระบบการจัดการการขนส่งแก๊ส ที่ครบในระบบเดียว
    \item ลดขั้นตอนในการดำเนินงานของพนักงาน
    \item การขนส่งแก๊สเป็นไปได้อย่างรวดเร็ว และมีประสิทธิภาพ
\end{enumerate}


\chapter{การวิเคราะห์ ออกแบบ และพัฒนาระบบ}
\label{chapter:problems}


\newpage
\section{Buisness Process เดิม}
\begin{figure}[h!]
    \centering
    \includegraphics[width=0.94\textwidth]{figure_business_process_old.pdf}
    \caption{แสดงข้อมูล Buisness Process เดิม}
    \label{fig:buisness_process_old}
\end{figure}

\subsection{คำอธิบาย Buisness Proess เดิม}
\begin{enumerate}
    \item เมื่อลูกค้าสนใจให้จัดส่งแก๊ส จะดำเนินการโทรสอบถามไปยังพนักงานจัดส่ง
    \item พนักงานจัดส่งตรวจสอบและยืนยันสถานะของรถขนส่ง และพนักงานขับรถ
    \item พนักงานจัดส่งวางแผนการจัดส่งแก๊สผ่านระบบไฟล์ Excel
    \item พนักงานจัดส่งดำเนินการจ่ายงานให้กับพนักงานขับรถคันต่าง ๆ ผ่านกลุ่ม LINE
    \item พนักงานขับรถต้องเตรียมความพร้อมก่อนที่จะทำการขนส่งแก๊สตามข้อปฎิบัติของบริษัทก่อนที่จะเริ่มงาน
    \item พนักงานขับรถกดรับงานผ่านกลุ่ม LINE
    \item พนักงานขับรถจัดส่งแก๊สตามจุดสถานที่ตามที่ได้รับมอบหมายจากพนักงานจัดส่ง \label{step:condition_old_flow_2}
    \item เมื่อพนักงานขับรถจัดส่งแก๊สตาที่ได้รับมอบหมายและพนักงานจัดส่งตรวจสอบแล้วพบว่ารถไปไม่ถึงจุดหมายตามเวลาที่กำหนด พนักงานขนส่งจะดำเนินการแจ้งลูกค้าให้รับทราบปัญหาที่เกิดขึ้นเพื่อเตรียมการรับมือ \label{step:condition_old_flow_1}
    \item พนักงานจัดส่งบันทึกประวัติการขนส่งลงในไฟล์ Excel ว่าพนักงานขนส่งถึงแต่ละจุดหมายกี่โมง และทำเวลาได้ตามที่กำหนดหรือไม่
    \item เมื่อพนักงานขนส่งขับรถถึงจุดหมายปลายทางลงแก๊สจะแจ้งอัปเดทเจ้าหน้่าที่ขนส่งผ่านกลุ่ม LINE ถ้าไม่กลับไปที่ \ref{step:condition_old_flow_1}.
    \item พนักงานขับรถกลับเดินทางไปยังจุดจ่ายงานถ้าไม่มีรายการขนส่งพ่วง (2 สถานที่จัดส่ง) ถ้ามีกลับไปที่ \ref{step:condition_old_flow_2}.
    \item พนักงานจัดส่งบันทึกปริมาณแก๊สที่ส่ง และที่เหลือจากพนักงานขนส่ง ลงในไฟล์ Excel
    \item พนักงานจัดส่งคำณวณค่าเที่ยวให้กับพนักงานขับรถ
    \item พนักงานจัดส่งอนุมัติค่าเที่ยวของพนักงานขับรถ
    \item พนักงานจัดส่งคำณวณค่าจ้างสุทธิเพื่อเรียกเก็บจากลูกค้า
    \item ลูกค้าจ่ายค่าจ้างให้กับทางบริษัท
    \item พนักงานจัดส่งนำข้อมูลทั้งหมดทำสรุปรายงาน เพื่อพัฒนาการขนส่งในครั้งต่อไป
\end{enumerate}

\subsection{ปัญหาของ Buisness Process เดิม}
\begin{enumerate}
    \item พนักงานจัดส่งต้องหางานที่ได้รับมอบหมาย และเรียกดูข้อมูลภายหลังด้วยตัวเองผ่านกลุ่ม LINE
    \item พนักงานขับรถต้องเรียกดูข้อมูลย้อนหลังด้วยตัวเองในกลุ่ม LINE
    \item พนักงานจัดส่งใช้เวลาในการติดตามสถานะของรถขนส่ง และต้องบันทึกข้อมูลเวลาด้วยมือ
    \item พนักงานจัดส่งต้องบันทึกปริมาณแก๊สแต่ละรอบด้วยมือ
    \item พนักงานจัดส่งต้องเก็บรวบรวมข้อมูลการขนส่งทั้งหมดสร้างเป็นรายงานด้วยตัวเอง
\end{enumerate}


\section{Buisness Process ใหม่}
\begin{figure}[h!]
    \centering
    \includegraphics[width=0.85\textwidth]{figure_buisness_process_new.pdf}
    \caption{แสดงข้อมูล Buisness Process ใหม่}
    \label{fig:buisness_process_new_main}
\end{figure}

\subsection{คำอธิบาย Buisness Process ใหม่}
\textbf{ส่วนของลูกค้า, พนักงานจัดส่ง และพนักงานขับรถ}
\begin{enumerate}
    \item ลูกค้าดำเนินการว่าจ้างให้ขนส่งผ่านพนักงานขนส่ง
    \item พนักงานจัดส่งยืนยันสถานะรถ และพนักงานขับรถเพื่อเตรียมความพร้อมในการจัดส่ง
    \item พนักงานจัดส่งวางแผนการจัดส่งแก๊สผ่านไฟล์ Excel
    \item พนักงานจัดส่งอัปโหลดไฟล์แผนงานเข้าสู่ระบบ Web Application \label{step:condition_new_flow_1}
    \item พนักงานจัดส่งตรวจสอบ format ของไฟล์แผนงาน
    \item ถ้าข้อมูลแผนการจัดส่งตรงตามรูปแบบ พนักงานขับรถกดรับงานในระบบผ้าน application ใน LINE OA ถ้าไม่กลับไปที่ \ref{step:condition_new_flow_1}.
    \item พนักงานขับรถต้องเตรียมความพร้อมก่อนที่จะทำการขนส่งแก๊สตามข้อปฎิบัติของบริษัทก่อนที่จะเริ่มงาน
    \item พนักงานขับรถกดเริ่มงานในระบบ Web Application ใน LINE OA
    \item พนักงานขับรถติดตามสถานะและตำแหน่งของรถแต่ละคัน\label{step:condition_new_flow_3}
    \item ถ้ารถขนส่งเกิดล่าช้่ากว่ากำหนด พนักงานจัดส่งต้องแจ้งลูกค้าให้รับทราบปัญหาที่เกิดขึ้นเพื่อเตรียมการรับมือ ถ้าไม่ให้ข้ามไปที่ \ref{step:condition_new_flow_2}.
    \item พนักงานจัดส่งว่าตรวจสอบว่ารถถึงจุดหมายปลายทางหรือยัง \label{step:condition_new_flow_2}
    \item ถ้าพนักงานจัดส่งตรวจสอบว่ารถถึงจุดหมายปลายทางแล้ว พนักงานขับรถจะบันทึกปริมาณแก๊สที่ส่ง และที่เหลือผ่านระบบ Web Application ใน LINE OA ถ้าไม่กลับไปที่  \ref{step:condition_new_flow_3}.
    \item ถ้ามีแก๊สพ่วง (2 สถานที่จัดส่ง) พนักงานขับรถกลับไปที่ \ref{step:condition_new_flow_4}. ถ้าไม่พนักงานขับรถเดินทางกลับไปยังจุดจ่ายงาน
    \item พนักงานจัดส่งตรวจสอบการกรอกกิโลกรัมแก๊ส
    \item ถ้ากรอกตรงตามจำนวนพ่วง พนักงานขับรถกดจบงานใน LINE OA ถ้าไม่กลับไปที่ \label{step:condition_new_flow_4}.
    \item พนักงานจัดส่งคำณวณค่าเที่ยวของ Order
    \item พนักงานจัดส่งตรวจสอบความถูกต้องของ Order
    \item ถ้าค่าเที่ยวไม่ถูกต้อง พนักงานจัดส่งแก้ไขค่าเที่ยว ถ้าไม่ข้ามไปที่ \ref{step:condition_new_flow_4}.
    \item พนักงานจัดส่งอนุมัติค่าเที่ยวให้พนักงานขับรถ \cite{step:condition_new_flow_4}
    \item พนักงานจัด่งนำออกราละเอียดค่าจ้างในแต่ละ Order
    \item พนักงานจัดส่งวางบิลค่าใช้จ่ายของผู้ว่าจ้าง
    \item ลูกค้าชำระค่าจ้างให้กับบริษัท
    \item พนักงานจัดส่งตรวจสอบผลการขนส่ง
\end{enumerate}

\chapter{โครงสร้างเอกสาร}
\label{chapter:DocumentStructure}

\section{โครงสร้างของแฟ้มต้นฉบับ}
ในเอกสารต้นฉบับสำหรับการเรียงพิมพ์ด้วย \LaTeX~นั้น นอกจากจะมีข้อความที่ต้องการให้แสดงผลแล้ว ยังมีคำสั่งพิเศษต่าง ๆ ที่ใช้ในการเรียงพิมพ์ด้วย
\begin{figure}[h]
\begin{lstlisting}
\documentclass[a4paper,twocolumn]{article}
% preamble
\usepackage{fontspec}

\title{My first article}
\author{My Name}

%document
\begin{document}
	\maketitle
	
	My text
\end{document}
\end{lstlisting}
	\caption{โครงสร้างแฟ้มต้นฉบับเอกสาร \LaTeX}
	\label{Fig:Structure}
\end{figure}

โดยทั่วไป แฟ้มต้นฉบับจะประกอบด้วยสองส่วนหลัก คือ ส่วนก่อนเริ่มต้นเอกสาร (preamble) และส่วนที่เป็นเอกสาร (document) ดังรูปที่~\ref{Fig:Structure}
โดย
\begin{itemize}
	\item ส่วนที่เป็นเอกสารคือทั้งหมดที่อยู่ระหว่าง \lstinline|\begin{document}| และ \lstinline|\end{document}| ซึ่งคือบรรทัดที่ 8-13
	\item และสิ่งที่อยู่ก่อนหน้านั้นทั้งหมด (บรรทัดที่ 1-7) เรียกว่า ส่วนก่อนเริ่มต้นเอกสาร
\end{itemize}

ส่วนก่อนเริ่มต้นเอกสารจะเป็นตัวกำหนดการตั้งค่าเอกสารที่ไม่ขึ้นกับเนื้อหาเฉพาะจุด เช่น แบบอักษร ตำแหน่งของหัวกระดาษและท้ายกระดาษ ขนาดกระดาษ ซึ่งค่าเหล่านี้ใช้ร่วมกันทั้งเอกสาร
และเป็นส่วนสำหรับประกาศการเรียกใช้แพ็คเกจต่าง ๆ ที่จะถูกเรียกใช้ในเอกสาร เพื่อให้โปรแกรมเรียงพิมพ์สามารถเรียกใช้ได้ถูกต้องเมื่อทำงาน

\section{การจัดหน้าทั่วไป}
\subsection{การใช้รูปแบบเอกสารต่าง ๆ ที่กำหนดไว้แล้ว}
ป้ายระบุ \lstinline|\documentclass| ที่บรรทัดแรกของเอกสารจะเป็นตัวกำหนดคลาส (class) หรือรูปแบบสำหรับการเรียงพิมพ์
คลาสมาตรฐานที่มีมาให้เมื่อลงโปรแกรมได้แก่ article, report, plain
แต่วารสารและการประชุมวิชาการต่าง ๆ รวมถึงวิทยานิพนธ์ของมหาวิทยาลัยจะกำหนดรูปแบบเฉพาะในการเรียงพิมพ์เอาไว้ เช่น ieeetran, amsart

รูปแบบการเรียงพิมพ์ เช่น การตั้งกั้นหน้า การขึ้นย่อหน้า ขนาดตัวอักษร รูปแบบตาราง การใส่ชื่อรูปภาพและตาราง จะกำหนดไว้ในแฟ้มที่มีนามสกุล cls (class) ซึ่งสิ่งตีพิมพ์ต่าง ๆ จะสร้างมาให้แล้วและให้ดาวน์โหลดไปใช้ได้
วิธีการใช้งานรูปแบบเหล่านี้ทำได้โดยกำหนดป้ายระบุ documentclass ให้เป็นชื่อรูปแบบแฟ้มที่มีนามสกุล cls นั้นๆ


\section{โครงสร้างเนื้อหาในเอกสาร}
ตอนต้นของส่วนเนื้อหา (document) มักจะเป็นข้อมูลชื่อเรื่อง (title) ชื่อผู้แต่ง (author) วันที่ (date) ของเอกสารนั้นๆ template ส่วนมากจะให้ผู้เขียนเอกสารระบุข้อมูลเหล่านี้ แล้วนำไปเรียงพิมพ์ให้ด้วยคำสั่ง \lstinline|\maketitle|

ถัดจากส่วนหัวข้อ มักจะเป็นส่วนบทคัดย่อ (abstract) ซึ่งระบุด้วย environment abstract

และหลังจากนั้น จึงเป็นเนื้อหาจริงๆ หากเป็นหนังสือหรือรายงานจะแบ่งเป็นบท (chapter) หากเป็นบทความวิจัย มักจะเริ่มต้นที่ตอน (section) และแยกย่อยลงไปเป็นตอนย่อย (subsection) และอาจะมีตอนย่อยลงไปอีก (subsubsection)
เราสามารถกำหนดข้อความต่างๆให้เป็นชื่อเรื่อง ชื่อผู้แต่ง ชื่อบท ชื่อตอน ได้ โดยใช้ป้ายระบุแบบกำหนดขอบเขต ดังนี้

\begin{multicols}{2}
	\begin{itemize}
		\item \lstinline|\title{...}| 
		\item \lstinline|\author{...}| 
		\item \lstinline|\chapter{...}| 
		\item \lstinline|\section{...}| 
		\item \lstinline|\subsection{...}| 
		\item \lstinline|\subsubsection{...}| 
	\end{itemize}
\end{multicols}
ลำดับเลขของส่วนต่าง ๆ จะถูกกำหนดขณะเรียงพิมพ์เรียงลำดับตามเนื้อหา ไม่จำเป็นต้องกำหนดตัวเลขเอง

สำหรับการทำสารบัญนั้น หากใช้โครงสร้างเนื้อหาด้วยป้ายระบุต่าง ๆ แล้ว  การสร้างสารบัญทำได้โดยป้ายระบุ \lstinline|\tableofcontents| รูปแบบของหน้าสารบัญถูกกำหนดโดยคลาส
ส่วนสารบัญภาพและตารางทำได้โดยเติมป้ายกำกับภาพและตารางทั้งหมด แล้วสร้างโดยใช้ป้ายระบุ \lstinline|\listoftables| และ \lstinline|\listoffigures| ตามลำดับ 

\section*{หัวข้อไม่นับเลข}
ทั้งนี้ เราสามารถระบุโครงสร้างแต่ไม่นับเลขได้ด้วยการใส่ * หลังป้ายกำกับคำสั่งระบุระดับ หัวข้อระดับนี้จะไม่ปรากฏในสารบัญ

จะเห็นได้ว่า หัวข้อนี้มีรูปแบบเดียวกับ section เว้นแต่ว่าไม่มีเลขลำดับ และหัวข้อถัดไป จะนับเลขต่อจากหัวข้อก่อนหน้า โดยนับข้ามหัวข้อนี้ไป

\section{โครงสร้างเนื้อหาในเอกสาร (ต่อ)}
\subsection{โครงสร้างหนังสือ}
หนังสือมักจะแบ่งส่วนเอกสารออกเป็นส่วนหน้า (frontmatter) ส่วนเนื้อหา (mainmatter) และส่วนหลัง (backmatter) การแบ่งเหล่านี้มักจะอำนวยความสะดวกในการนับเลขหน้าเพื่อไปใส่ในสารบัญด้วย โดยส่วนหน้าและส่วนหลังมักจะไม่นับเลขหน้า หรือหากนับ ก็จะนับแยกจากส่วนเนื้อหา และใช้เลขโรมัน

ข้อมูลที่มักจะอยู่ในส่วนหน้าของหนังสือ ได้แก่
\begin{enumerate}
	\item หน้าปก
	\item คำนำ (preface)
	\item คำอุทิศ (acknowledgement)
	\item สารบัญ (tableofcontents)
	\item สารบัญรูป (listoffigures)
	\item สารบัญตาราง (listoftables)
\end{enumerate}
การเรียงลำดับข้อมูลนี้สามารถปรับได้ตามความต้องการ เช่นในหนังสือหรือตำราเรียน มักจะใส่คำนำไว้หลังจากสารบัญ เป็นต้น

ถัดจากส่วนเนื้อหา สามารถมีส่วนภาคผนวก ก่อนจะเป็นส่วนท้าย ซึ่งเป็นบรรณานุกรม ดรรชนี และข้อมูลเกี่ยวกับผู้เขียน


\chapter{การจัดรูปแบบ}

\section{ขนาดและรูปแบบตัวอักษร}
\label{Subsect:Font}
\begin{table}[h]
	\center
	\caption{การกำหนดขนาดของตัวอักษร}
	\begin{tabular}{|c|c|}
		\hline
		คำสั่ง	& ผลลัพธ์ \\
		\hline
		\lstinline|\tiny|	& \tiny text ข้อความ  \\
		\lstinline|\scriptsize|	& \scriptsize text ข้อความ \\
		\lstinline|\footnotesize|	& \footnotesize text ข้อความ \\
		\lstinline|\small|	& \small text ข้อความ \\
		\lstinline|\normalsize|	& {\normalsize text ข้อความ} ขนาดปกติ \\
		\lstinline|\large|	& \large text ข้อความ \\
		\lstinline|\Large|	& \Large text ข้อความ \\
		\lstinline|\huge|	& \huge text ข้อความ \\
		\lstinline|\Huge|	& \Huge text ข้อความ \\
		\hline
	\end{tabular}
	\label{Table:FontSize}
\end{table}

ในการพิมพ์ข้อความอาจมีความจำเป็นต้อง {\Large เพิ่ม} หรือ {\scriptsize ลด} ขนาดของตัวอักษรในบางจุด การสร้างเอกสารโดย \LaTeX~จะมีป้ายระบุเพื่อใช้กำหนดขนาดสัมพัทธ์กับขนาดปกติโดยใช้ป้ายระบุดังแสดงในตารางที่ \ref{Table:FontSize}
โดยที่ป้ายระบุเพื่อกำหนดขนาดตัวอักษรเหล่านี้เป็นป้ายระบุแบบไม่ใช้ตัวแปร ซึ่งจะเรียงพิมพ์ด้วยคำสั่งนี้ไปจนกว่าจะหมดขอบเขตของข้อความ หากเราต้องการจำกัดขอบเขตของป้ายระบุสามารถทำได้โดยใส่วงเล็บปีกกาครอบทั้งป้ายระบุและข้อความเพื่อกำหนดขอบเขตบล็อกที่จะให้ป้ายระบุนั้นมีผลในการเรียงพิมพ์

\begin{table}
	\center
	\caption{การจัดรูปแบบตัวอักษร}
	\begin{tabular}{|c|c|}
		\hline
		คำสั่ง		& ผลลัพธ์ \\
		\hline
		\lstinline|\textit{...}| 	& \textit{ตัวเอียง} \\
		\lstinline|\textbf{...}| 	& \textbf{ตัวหนา} \\
		\lstinline|\underline{...}| 	& \underline{ขีดเส้นใต้} \\
		\lstinline|\sout{...}| 	& \sout{ขีดกลาง} \\
		\hline
	\end{tabular}
	\label{Table:TextFormat}
\end{table}

\textbf{การทำตัวหนา} \textit{ตัวเอียง} \underline{ขีดเส้นใต้} \sout{ขีดกลาง} (สำหรับขีดกลางต้องเรียกใช้แพ็คเกจ ulem) ใช้ป้ายระบุดังที่สรุปอยู่ในตาราง \ref{Table:TextFormat} การจัดรูปแบบตัวอักษร\textbf{\underline{\textit{ผสม}}} กันหลายรูปแบบสามารถทำได้ด้วยการใช้ป้ายระบุซ้อนกัน

โดยปกติเอกสาร \LaTeX~จะใช้ฟอนต์เดียวกันทั้งเอกสาร หากไม่ได้กำหนดฟอนต์เป็นพิเศษ ตัวอักษรทั้งหมดจะใช้ฟอนต์ Computer Modern
การเปลี่ยนฟอนต์เฉพาะบางส่วนของเอกสารทำได้หลายวิธี หากใช้ Xe\TeX/Xe\LaTeX~ซึ่งสนับสนุน Unicode การกำหนดฟอนต์จะใช้แพ็คเกจ fontspec แล้วระบุตัวแปรเป็น font ที่ต้องการ เช่น {\fontspec{TeX Gyre Heros}TeX Gyre Heros} หรือ {\fontspec{TeX Gyre Cursor}TeX Gyre Heros}
ฟอนต์ที่สามารถใช้ได้คือ True Type Font หรือ OpenType font ใด ๆ ที่ลงไว้ในเครื่อง สำหรับฟอนต์มาตรฐานของเอกสารราชการไทยในปัจจุบัน (พ.ศ. 2557) คือ TH Sarabun New 
การกำหนดฟอนต์หลักจึงทำได้ดังนี้
\clearpage
\begin{lstlisting}[numbers=none]
\usepackage{fontspec}
\setmainfont{TH Sarabun New}
\end{lstlisting}

เอกสารนี้ใช้คลาส memoir และกำหนดฟอนต์หลักเป็น TeX Gyre Thermes เทียบเท่า Times New Roman ใน Windows ขนาดของฟอนต์ปกติ 12 pt
และกำหนดขนาดฟอนต์ภาษาไทยให้ขนาดอักษรปกติเทียบเท่ากับอักษรตัวพิมพ์ใหญ่ในภาษาอังกฤษ

\section{การจัดเรียงข้อความ}
หากไม่ได้ตั้งค่าใดเป็นพิเศษ ข้อความจะถูกจัดให้ชิดทั้งซ้ายและขวาเสมอ (fully justified)
\begin{center}การจัดข้อความให้อยู่กึ่งกลางพื้นที่\end{center}
\begin{flushright}ชิดขวา\end{flushright}
\begin{flushleft}หรือบังคับให้ชิดซ้าย\end{flushleft}
ทำได้โดยคำสั่งต่อไปนี้
\begin{itemize}
	\item กึ่งกลาง
	\begin{itemize}[label=$\circ$]
		\item \lstinline|\centering| หรือ
		\item \lstinline|\begin{center} ... \end{center}|
	\end{itemize}
	\item ชิดขวา
	\begin{itemize}[label=$\circ$]
		\item \lstinline|\raggedleft| หรือ
		\item \lstinline|\begin{flushright}...\end{flushright}|
	\end{itemize}
	\item ชิดซ้าย
	\begin{itemize}[label=$\circ$]
		\item \lstinline|\raggedright| หรือ
		\item \lstinline|\begin{flushleft}...\end{flushleft}|
	\end{itemize}
\end{itemize}

ปัญหาหนึ่งที่พบกับการใช้ \LaTeX~ในภาษาไทยคือการตัดคำเพื่อจัดหน้าทำได้ไม่ดีนัก บางกรณีอาจมีคำที่ล้นขอบขวาเนื่องจากโปรแกรมไม่สามารถตัดคำได้ ปัญหานี้ยังไม่มีวิธีการแก้ไขที่ถาวร ผู้ใช้จำเป็นต้องเลือกเว้นวรรคข้อความ หรือปรับเนื้อหาข้อความเองเพื่อให้โปรแกรมพยายามเรียงพิมพ์ได้อย่างเหมาะสม

การจัดชิดซ้ายและขวาในภาษาอังกฤษนั้น \LaTeX~สามารถตัดกึ่งกลางคำโดยใช้ - (hyphen) ได้โดยอัตโนมัติ และทำได้ค่อนข้างดี ซึ่งปกติแล้วการเว้นช่องไฟต่าง ๆ เพื่อจัดหน้าในภาษาอังกฤษด้วย \LaTeX~จะสวยงามกว่าการใช้ Microsoft Word

\section{ช่องว่างแบบต่างๆ}
การเคาะวรรค ย่อหน้า หรือขึ้นบรรทัดใหม่หนึ่งบรรทัดใน \LaTeX~ นั้นมีผลเหมือนกันในการเรียงพิมพ์ คือ อนุญาตให้จัดช่องไฟกว้างแคบได้ตามความเหมาะสมเมื่อต้องการตัดคำเพื่อขึ้นบรรทัดใหม่
การเคาะวรรคหรือย่อหน้าในการพิมพ์เอกสารจึงมีไว้เพื่อให้ผู้พิมพ์อ่านข้อความได้ง่ายเท่านั้น
นอกจากนี้ การขึ้นบรรทัดใหม่บรรทัดเดียวมักมีประโยชน์ในการเขียนย่อหน้ายาวๆ
หากเป็นข้อความภาษาอังกฤษ บางครั้งจะนิยมเขียนข้อความบรรทัดละประโยค เพื่อให้สะดวกในการกลับมาหาข้อความเพื่อแก้ไขในอนาคต

แต่การจัดช่องไฟอัตโนมัติอาจทำให้การแสดงผลบางรูปแบบไม่ถูกต้อง เช่น หลังคำว่ารูปที่ ควรเคาะเพียงวรรคเดียวแล้วตามด้วยตัวเลข การบังคับให้เครื่องเว้นวรรคเดียวและห้ามตัดระหว่างคำทำได้โดยใช้เครื่องหมาย {\textasciitilde} แทนการเคาะวรรค เช่น รูปที่~1 ส่วนการบังคับให้ขึ้นบรรทัดใหม่ ใช้คำสั่ง \lstinline+\\+ \\
หรือใช้ป้ายระบุ \lstinline|\newline| \newline
ก็ได้ ทุกครั้งที่ขึ้นบรรทัดใหม่โดยใช้คำสั่งหรือป้ายระบุนี้จะไม่มีการย่อหน้า
หากต้องการให้ย่อหน้าใหม่ให้ใช้การขึ้นบรรทัดใหม่สองครั้ง \\
\indent หรือใช้ป้ายระบุ \lstinline|\indent| เพื่อบังคับให้ขึ้นย่อหน้าใหม่

\begin{comment}
	ระยะระหว่างบรรทัดจะกำหนดไว้ในคลาส แต่หากต้องการระบุเอง มีชุดคำสั่งในแพ็คเกจ setspace
	ได้แก่ \lstinline|\singlespacing|, \lstinline|\onehalfspacing|, \lstinline|\doublespacing|
	สำหรับระยะระหว่างบรรทัดหนึ่งเท่า หนึ่งเท่าครึ่ง และสองเท่า ตามลำดับ
	หากต้องการกำหนดเอง ใช้ป้ายระบุ \lstinline|\setstretch{<factor>}| แล้วกำหนด factor เป็นตัวเลขที่ต้องการ
	ในกรณีที่ไม่ต้องการใช้แพ็คเกจ อาจใช้ป้ายระบุ \lstinline|\linespread{<factor>}| ก็ได้ ซึ่งมีผลเหมือน \lstinline|\setstretch{<factor>}|
\end{comment}

บางครั้งอาจต้องบังคับเว้นระยะระหว่างย่อหน้า ระหว่างข้อความ รูปภาพ มากกว่าค่าโดยปริยายที่รูปแบบเอกสารจัดให้ คำสั่งที่ใช้ในการเว้นระยะแนวนอนคือ
\begin{lstlisting}[numbers=none]
	\hspace{<length>}
\end{lstlisting}
เช่น \hspace{10cm} อยู่ห่างจากคำก่อนหน้า 10 ซม. การกำหนดระยะสามารถระบุเป็น ซม. (cm) นิ้ว (in) พอยท์ (pt) ได้

\vspace{-1.5cm}
ส่วนคำสั่งที่ใช้ในการเว้นระยะแนวตั้งคือ
\begin{lstlisting}[numbers=none]
	\vspace{<length>}
\end{lstlisting}
จะเห็นว่าระยะที่ใช้จะเป็นลบหรือบวกก็ได้เช่นกัน


\section{ข้อย่อย}
โครงสร้างเนื้อหาอีกรูปแบบซึ่งใช้บ่อยคือข้อย่อย ใน \LaTeX~ มีโครงสร้างข้อย่อยให้เลือก 2 รูปแบบ แบบนับเลขข้อ (enumerate) และแบบไม่นับเลขข้อ (itemize) ตัวข้อย่อยแต่ละตัวจะอยู่หลังป้ายระบุ \lstinline|\item| หากต้องการซ้อนข้อย่อยก็แทรกโครงสร้างข้อย่อยลงไปในป้ายระบุ item ได้ เช่น
\begin{enumerate}
	\item ข้อแรกไม่มีข้อย่อย
	\item ข้อสองแยกเป็นข้อย่อยแบบไม่นับเลขสองข้อ ได้แก่
	\begin{itemize}
		\item bullet แรก
		\item bullet ที่สอง
	\end{itemize}
	และแบบนับเลขข้ออีกสามข้อ ดังนี้
	\begin{enumerate}
		\item ข้อสอง ข้อย่อยที่ 1
		\item ข้อสอง ข้อย่อยที่ 2
		\item ข้อสอง ข้อย่อยที่ 3
	\end{enumerate}
	\item ข้อสาม
\end{enumerate}
ทั้งนี้ เราสามารถกำหนดวิธีการนับเลข (เช่น เลขไทย เลขโรมัน อักขระ) และรูปสัญลักษณ์ของ bullet (วงกลมทึบ วงกลมโปร่ง สี่เหลี่ยม หรือสัญลักษณ์อื่นๆ) ได้เองผ่านตัวเลือกต่าง ๆ ผู้ที่สนใจสามารถไปค้นหาวิธีเพิ่มเติมได้ในอินเทอร์เน็ต

\chapter{การแทรกรูปภาพและตาราง}
\section{การแทรกรูปภาพ}
การแทรกไฟล์รูปภาพทำได้โดยใช้แพ็คเกจ graphicx และป้ายระบุดังนี้
\begin{lstlisting}[numbers=none]
\includegraphics[options]{filename}
\end{lstlisting}

\begin{figure}[h]
	\begin{subfigure}{.5\textwidth}
		\includegraphics{Figures/Prakeaw-Pink}
		
		\lstinline|\includegraphics{Figures/Prakeaw-Pink}|
		\subcaption{รูปเต็ม}
	\end{subfigure}
	\begin{subfigure}{.5\textwidth}
		\centering	% กลางพื้นที่
		\includegraphics[scale=.5]{Figures/Prakeaw-Pink}
		
		\lstinline|\includegraphics[scale=.5]{Figures/Prakeaw-Pink}|
		\subcaption{รูปขนาดครึ่งของรูปเต็ม}
	\end{subfigure}
	
	\begin{subfigure}{.5\textwidth}
		\includegraphics[width=1cm]{Figures/Prakeaw-Pink}
		
		\lstinline|\includegraphics[width=1cm]{Figures/Prakeaw-Pink}|
		\subcaption{รูปขนาดกว้าง 1 ซม. ปรับความสูงอัตโนมัติ}
		\label{Fig:Fig1cm}
	\end{subfigure}
	\begin{subfigure}{.5\textwidth}
		\raggedleft	% ชิดขวา
		\subcaption{รูปขนาดกว้าง 2 ซม. ความสูง 1 ซม.}
		\includegraphics[width=2cm,height=1cm]{Figures/Prakeaw-Pink}
		
		\lstinline|\includegraphics[width=2cm,height=1cm]{Figures/Prakeaw-Pink}|
		\label{Fig:w2h1}
	\end{subfigure}
	\caption{ตัวอย่างการแทรกรูปภาพ}
	\label{Fig:Fig}
\end{figure}

คำสั่ง includegraphics มีตัวเลือกเพื่อปรับขนาดของภาพที่จะแทรกได้ด้วย ดังตัวอย่างในรูป~\ref{Fig:Fig} โดยปกติเรามักใช้ป้ายระบุ figure แบบเปิด/ปิดเพื่อกำหนดขอบเขตของรูป ภายในเรียกใช้คำสั่ง includegrapics แล้วตามด้วยป้ายกำกับชื่อรูปภาพเหล่านั้น เพื่อให้โปรแกรมเรียงพิมพ์สามารถแยกแยะได้ว่าป้ายกำกับเหล่านี้เป็นป้ายกำกับของรูปภาพ และสามารถนับลำดับเฉพาะรูปภาพได้

ส่วนคำบรรยายภาพจะถูกกำหนดโดยป้ายระบุ \lstinline|\caption{..}| โดยมากแล้ววารสารต่างๆ มักให้พิมพ์ข้อความบรรยายภาพไว้ใต้ภาพ แต่หากต้องการให้ข้อความบรรยายภาพอยู่เหนือภาพก็ทำได้โดยย้ายตำแหน่งของป้ายระบุ caption ไปไว้ก่อนภาพ ดังรูป~\ref{Fig:w2h1}

หากรูปมีขนาดเล็กกว่าพื้นที่หน้ากระดาษ โดยปกติรูปจะถูกวางไว้ชิดซ้าย หากต้องการจัดให้รูปอยู่กึ่งกลางหรือชิดขวาของพื้นที่ก็ใช้วิธีเช่นเดียวกับการจัดหน้าปกติ นอกจากนี้ หากต้องการใส่รูปย่อยหลายรูปในภาพเดียวกัน เราสามารถวางรูปด้วยป้ายระบุ includegraphics ต่อ ๆ กันไปได้เลย แต่หากต้องการใส่คำอธิบายภาพแยกตามรูปย่อย สามารถทำได้ผ่านการแทรกรูปย่อยด้วยแพ็คเกจ subcaption ซึ่งใช้ป้ายระบุ subfigure ที่มีโครงสร้างเหมือนป้ายระบุ figure ทุกประการ การอ้างอิงถึงรูปย่อยก็ทำเช่นเดียวกับการอ้างอิงถึงรูปปกติ เช่น รูปที่~\ref{Fig:Fig1cm} อ้างอิงถึงรูปขนาดกว้าง 1 ซม. ปรับความสูงอัตโนมัติ เป็นต้น

โปรแกรมเรียงพิมพ์จะจัดตำแหน่งของรูปภาพในเอกสารให้เอง ค่าเริ่มต้นที่ตั้งมามักเป็นการจัดให้รูปภาพอยู่บนสุดของหน้ากระดาษ แต่หากต้องการกำหนดตำแหน่งของรูปภาพให้อยู่ท้ายกระดาษ หรือ ณ บริเวณที่กำหนด เราสามารถระบุได้ผ่านตัวเลือกของป้ายระบุ figure ได้ เช่น

\begin{tabular}{lc}
	h    & ณ ตำแหน่งใกล้ ๆ นี้\\
	H    & ณ ตำแหน่งนี้ (ต้องใช้แพ็คเกจ float) \\
	t    & ด้านบนของกระดาษ \\
	b    & ด้านล่างของกระดาษ
\end{tabular}

หากไม่ใช้ป้ายระบุ figure ครอบรูปภาพ โปรแกรมจะเรียงพิมพ์รูปนั้นเสมือนเป็นอักขระหนึ่ง เช่น \includegraphics[width=.05\textwidth]{Figures/Prakeaw-Black}

\section{การสร้างตาราง}
โครงสร้างตารางอย่างง่ายใช้ป้ายระบุ tabular ดังตัวอย่างต่อไปนี้
\begin{lstlisting}[numbers=none]
\begin{tabular}{|c||lr|}
t1	& t2		& t3 \\
\hline
abcd	& defghij	& klmnop \\
\hline
\end{tabular}
\end{lstlisting}
เมื่อเรียงพิมพ์แล้ว จะให้ผลออกมาเป็น

\begin{tabular}{|c||lr|}
	t1		& t2	& t3 \\
	\hline
	abcd	& defghij	& klmnop \\
	\hline
\end{tabular}

วงเล็บปีกกาแรกถัดจาก tabular คือจำนวนและการจัดหน้าของแต่ละคอลัมน์ จากตัวอย่างข้างต้นนี้มีทั้งหมด 3 คอลัมน์ โดย c l r แทนกึ่งกลาง ชิดซ้าย และชิดขวา ตามลำดับ ตัวอักษร | แทนการขีดเส้นตั้งระหว่างคอลัมน์ ข้อความที่อยู่ระหว่าง begin และ end คือข้อมูลในตาราง ข้อความแต่ละบรรทัดคือข้อความในแต่ละแถวของตาราง คั่นระหว่างคอลัมน์ในแถวด้วยสัญลักษณ์ \& และจบแถวด้วยการสั่งขึ้นบรรทัดใหม่ การขีดเส้นนอนในตารางทำได้โดยคำสั่ง \lstinline|\hline|

ในทำนองเดียวกับรูปภาพ เรามักใช้ป้ายระบุ table แบบกำหนดขอบเขต เพื่อระบุขอบเขตของตาราง และให้โปรแกรมจัดตำแหน่งที่เหมาะสมสำหรับตารางให้เอง (หรือระบุตำแหน่งเองแบบเดียวกับรูปภาพ) รวมถึงนับลำดับการอ้างอิงแยกเป็นตารางด้วย

\chapter{การอ้างอิง}
\section{การอ้างอิงภายในเอกสาร}
ป้ายกำกับ (label) มีโครงสร้างดังนี้
\begin{lstlisting}[numbers=none]
\label{<label name>}
\end{lstlisting}
โดย \lstinline|<label name>| คือชื่อของป้ายกำกับ การใส่ป้ายกำกับลงไปในเอกสารเพื่อใช้อ้างถึงตำแหน่งนั้น ๆ การอ้างอิงถึงบทและตอน (และตอนย่อยหากมี) ของตำแหน่งป้ายกำกับนั้นใช้ป้ายระบุ \lstinline|\ref<label name>| เช่น หากกำหนดป้ายกำกับที่นี่ \lstinline|\label{sample}| \label{sample} การอ้างถึง \lstinline|\ref{sample}| จะได้เป็น \ref{sample} ส่วนการอ้างถึงหน้าที่ตำแหน่งป้ายกำกับนั้นใช้คำสั่ง \lstinline|\pageref{<label name>}| เช่น \lstinline|\pageref{sample}| จะได้ \pageref{sample} และหากป้ายกำกับอยู่ในขอบเขตของรูปภาพและตาราง การอ้างถึงก็จะได้เลขลำดับของรูปภาพและตารางนั้นมา เช่น รูปที่ \ref{Fig:bibtex} อยู่ที่หน้า \pageref{Fig:bibtex} เป็นต้น

\section{บรรณานุกรม}
\begin{figure}
	\begin{verbatim}
@inproceedings{DBLP:conf/pricai/WanvarieTO10,
	author    = {Dittaya Wanvarie and
		Hiroya Takamura and
		Manabu Okumura},
	title     = {Active Learning for Sequence Labelling
		with Probability Re-estimation},
	booktitle = {PRICAI},
	year      = {2010},
	pages     = {681-686},
	ee = {http://dx.doi.org/10.1007/978-3-642-15246-7_69},
	crossref  = {DBLP:conf/pricai/2010},
	bibsource = {DBLP, http://dblp.uni-trier.de}
}

@book{thaibib,
	author = {{ฑิตยา หวานวารี}},
	title = {การใช้ \LaTeX สําหรับเรียงพิมพ์วิทยานิพนธ์ภาษาไทยและภาษาอังกฤษ โดยใช้รูปแบบของจุฬาลงกรณ์มหาวิทยาลัย},
	publisher = {{ภาควิชาคณิตศาสตร์และวิทยาการคอมพิวเตอร์}},
	month = {กรกฎาคม},
	year = {2557},
}
	\end{verbatim}
	\caption{ตัวอย่างข้อมูลสำหรับ BibTeX}
	\label{Fig:bibtex}
\end{figure}
Bib\TeX~เป็นโปรแกรมการจัดการรูปแบบเอกสารอ้างอิงที่ทำให้ \LaTeX~ สามารถเรียงพิมพ์ได้สะดวก หากดาวน์โหลดข้อมูลบรรณานุกรมจากเว็บของสำนักพิมพ์ ทางสำนักพิมพ์มักจะอำนวยความสะดวกด้วยการทำรายละเอียดสำหรับ Bib\TeX~ สำหรับบทความแต่ละชิ้นไว้ให้ สามารถคัดลอกไปใช้ได้ทันที เช่น รูปที่~\ref{Fig:bibtex}

ทั้งนี้ ควรตรวจแก้ไขสอบข้อมูลต่าง ๆ ให้ถูกต้องตามรูปแบบที่สิ่งพิมพ์แต่ละชนิดกำหนดด้วย เช่น ชื่อหนังสือ บางวารสารอาจบังคับให้ใช้ชื่อเต็มสำหรับการประชุมวิชาการ ในขณะที่ booktitle ในข้อมูลที่ได้มาเป็นชื่อย่อ ก็ต้องแก้เองให้ถูกต้องด้วย เป็นต้น

%อย่างไรก็ตาม Bib\TeX~ นั้นเป็นโปรแกรมเก่า ซึ่งจะมีปัญหาในการตัดบรรทัดเมื่อใช้กับข้อความที่ไม่ได้ใช้อักขระภาษาอังกฤษ ในเอกสารแก้ปัญหาแบบชั่วคราวด้วยการใช้ \% นำหน้าช่องว่าง เพื่อบอกขอบเขตของคำ ให้โปรแกรมสามารถตัดคำเพื่อขึ้นบรรทัดใหม่ได้อย่างถูกต้อง

การอ้างอิงถึงเอกสารชิ้นต่าง ๆ ในแฟ้มต้นฉบับทำได้โดยคำสั่ง
\begin{lstlisting}[numbers=none]
\cite{<cite name>}
\end{lstlisting}

และเขียนข้อมูลบรรณานุกรมของแต่ละเอกสารอ้างอิงให้ถูกต้องรวมกับไว้ในแฟ้มรายการบรรณานุกรม ซึ่งมักให้มีนามสกุล bib การระบุรูปแบบของบรรณานุให้ใช้รูปแบบที่วารสารกำหนดให้ใช้แฟ้มรูปแบบบรรณานุกรม (นามสกุล bst)

การระบุรูปแบบและแฟ้มรายการบรรณานุกรมอ้างอิงทำได้ดังนี้
\begin{lstlisting}[numbers=none]
\bibliographystyle{<bib style name>}
\bibliography{<bib name>}
\end{lstlisting}

การใช้ Bib\TeX~นั้น หากใช้โปรแกรมจำพวก IDE ในการสร้างเอกสารต้นฉบับ มักจะมีการเรียก Bib\TeX~ให้โดยอัตโนมัติ อย่างไรก็ตาม หากสั่งเรียงพิมพ์โดยตรง ต้องสั่งเรียงพิมพ์ก่อนหนึ่งครั้งเพื่อสร้างรายการบรรณานุกรม แล้วใช้ Bib\TeX~ เพื่อโยงรายการอ้างอิงเหล่านั้นกับแฟ้มข้อมูลรายการบรรณานุกรม แล้วจึงสั่งเรียงพิมพ์อีกรอบ เพื่อให้แทนค่าแต่ละตำแหน่งของคำสั่ง \textbackslash cite ด้วยข้อมูลบรรณานุกรมในเอกสารผลลัพธ์ให้ถูกต้อง

เช่นจากรูปที่~\ref{Fig:bibtex} สามารถอ้างอิงถึงได้ดังนี้
\begin{lstlisting}[numbers=none]
\cite{DBLP:conf/pricai/WanvarieTO10}
\end{lstlisting}
และผลลัพธ์ที่ได้เป็นดังนี้ \cite{DBLP:conf/pricai/WanvarieTO10} และ \lstinline|\citep{thaibib}| จะได้ผลลัพธ์เป็น \citep{thaibib}

\section{เอกสารสำหรับอ่านเพิ่มเติม}
รายละเอียดการใช้ \LaTeX~สามารถอ่านเพิ่มเติมได้ที่
\begin{itemize}
	\item \LaTeX~User's Guide and Reference Manual 
	
	\url{http://latex-project.org/guides/usrguide.pdf}
	\item \url{http://www.ctan.org/tex-archive/info/lshort/thai}
	\item \url{http://en.wikibooks.org/wiki/LaTeX}
	\item สัญลักษณ์ต่าง ๆ
	
	\url{http://www.tex.ac.uk/tex-archive/info/symbols/comprehensive/symbols-a4.pdf}
	
	\item ตัวอย่างบทความเมื่อใช้คลาสของ AMS
	
	\url{http://www.ams.org/publications/authors/tex/amslatex}
	
\end{itemize}
หรือค้นคว้าเพิ่มเติมได้จากอินเทอร์เน็ต

\chapter{การแทรกสมการและสัญลักษณ์คณิตศาสตร์}
การพิมพ์ข้อความหรือสัญลักษณ์คณิตศาสตร์แบ่งออกเป็น 2 รูปแบบ คือ แทรกในบรรทัดเดียวกัน (inline) เช่น $\delta^2$ หรือแยกออกมาเป็นบรรทัดต่างหากในกรณีพิมพ์สมการ เช่น
\begin{equation}
	f(x,y) = \int_a^b \frac{p(x,y)}{\ln y} dx \label{Eq:fxy}
\end{equation}
จะสังเกตได้ว่าการเรียงพิมพ์ในสภาพแวดล้อมคณิตศาสตร์จะใช้ฟอนต์ที่ต่างไปจากการเรียงพิมพ์ในสภาพแวดล้อมข้อความทั่วไป เมื่อใช้ฟังก์ชันหรือสัญลักษณ์ทางคณิตศาสตร์จึงควรใช้สภาพแวดล้อมคณิตศาสตร์เสมอ

การพิมพ์ในบรรทัดเดียวกันทำได้ด้วยการเขียนข้อความที่ต้องการให้อยู่ระหว่าง \$ และ \$ ส่วนการพิมพ์แยกบรรทัดต่างหากทำได้โดยการพิมพ์ข้อความให้อยู่ระหว่างป้ายระบุแบบเปิดปิด equation หรือ align

align จะสามารถเขียนสัญลักษณ์ต่าง ๆ ได้มากกว่า 1 บรรทัด มีได้หลายคอลัมน์ และสามารจัดเรียงตำแหน่งได้ด้วย เช่น
\begin{align}
	x    &= 2 \label{Eq:x} \\
	y    &= 3 \label{Eq:y} \\
	z    &= x \times y \nonumber \\
	&= p \label{Eq:result}s
\end{align}

การเรียงพิมพ์สมการแยกบรรทัดนั้น โปรแกรมจะสร้างหมายเลขกำกับสมการให้โดยอัตโนมัติ หากต้องการอ้างอิงถึงเลขสมการ ทำได้โดยเติมป้ายกำกับ (label) ลงไป ตั้งชื่อได้ตามต้องการ เมื่อต้องการเรียกใช้ก็เรียกใช้ป้ายกำกับนั้นๆ โดยใช้ป้ายระบุ \lstinline|\ref{<label name>}| เช่น เราอ้างอิงถึงสมการข้างต้นได้ด้วย (\ref{Eq:fxy}) คำแนะนำในการตั้งชื่อป้ายระบุคือควรตั้งให้สื่อถึงข้อความส่วนนั้น
หากไม่ต้องการให้มีเลขสมการ ให้เติม * ลงไปข้างหลัง equation เช่น

\begin{equation*}
	\mathbf{X}_k = \sum_{n=0}^{N-1} x_n \cdot e^{-i 2 \pi k n / N}, k \in \mathbb{Z}
\end{equation*}

การเขียนเมตริกซ์ การกำหนดนิยาม ทฤษฎีบท และอื่นๆ สามารถค้นคว้าเพิ่มเติมได้จากอินเทอร์เน็ต

\backmatter
\bibliographystyle{alpha}
\bibliography{MathCS}   
\end{document}